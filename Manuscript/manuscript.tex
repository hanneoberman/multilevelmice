% Options for packages loaded elsewhere
\PassOptionsToPackage{unicode}{hyperref}
\PassOptionsToPackage{hyphens}{url}
\PassOptionsToPackage{dvipsnames,svgnames,x11names}{xcolor}
%
\documentclass[
  article]{jss}

\usepackage{amsmath,amssymb}
\usepackage{lmodern}
\usepackage{iftex}
\ifPDFTeX
  \usepackage[T1]{fontenc}
  \usepackage[utf8]{inputenc}
  \usepackage{textcomp} % provide euro and other symbols
\else % if luatex or xetex
  \usepackage{unicode-math}
  \defaultfontfeatures{Scale=MatchLowercase}
  \defaultfontfeatures[\rmfamily]{Ligatures=TeX,Scale=1}
\fi
% Use upquote if available, for straight quotes in verbatim environments
\IfFileExists{upquote.sty}{\usepackage{upquote}}{}
\IfFileExists{microtype.sty}{% use microtype if available
  \usepackage[]{microtype}
  \UseMicrotypeSet[protrusion]{basicmath} % disable protrusion for tt fonts
}{}
\makeatletter
\@ifundefined{KOMAClassName}{% if non-KOMA class
  \IfFileExists{parskip.sty}{%
    \usepackage{parskip}
  }{% else
    \setlength{\parindent}{0pt}
    \setlength{\parskip}{6pt plus 2pt minus 1pt}}
}{% if KOMA class
  \KOMAoptions{parskip=half}}
\makeatother
\usepackage{xcolor}
\setlength{\emergencystretch}{3em} % prevent overfull lines
\setcounter{secnumdepth}{-\maxdimen} % remove section numbering
% Make \paragraph and \subparagraph free-standing
\ifx\paragraph\undefined\else
  \let\oldparagraph\paragraph
  \renewcommand{\paragraph}[1]{\oldparagraph{#1}\mbox{}}
\fi
\ifx\subparagraph\undefined\else
  \let\oldsubparagraph\subparagraph
  \renewcommand{\subparagraph}[1]{\oldsubparagraph{#1}\mbox{}}
\fi


\providecommand{\tightlist}{%
  \setlength{\itemsep}{0pt}\setlength{\parskip}{0pt}}\usepackage{longtable,booktabs,array}
\usepackage{calc} % for calculating minipage widths
% Correct order of tables after \paragraph or \subparagraph
\usepackage{etoolbox}
\makeatletter
\patchcmd\longtable{\par}{\if@noskipsec\mbox{}\fi\par}{}{}
\makeatother
% Allow footnotes in longtable head/foot
\IfFileExists{footnotehyper.sty}{\usepackage{footnotehyper}}{\usepackage{footnote}}
\makesavenoteenv{longtable}
\usepackage{graphicx}
\makeatletter
\def\maxwidth{\ifdim\Gin@nat@width>\linewidth\linewidth\else\Gin@nat@width\fi}
\def\maxheight{\ifdim\Gin@nat@height>\textheight\textheight\else\Gin@nat@height\fi}
\makeatother
% Scale images if necessary, so that they will not overflow the page
% margins by default, and it is still possible to overwrite the defaults
% using explicit options in \includegraphics[width, height, ...]{}
\setkeys{Gin}{width=\maxwidth,height=\maxheight,keepaspectratio}
% Set default figure placement to htbp
\makeatletter
\def\fps@figure{htbp}
\makeatother
\newlength{\cslhangindent}
\setlength{\cslhangindent}{1.5em}
\newlength{\csllabelwidth}
\setlength{\csllabelwidth}{3em}
\newlength{\cslentryspacingunit} % times entry-spacing
\setlength{\cslentryspacingunit}{\parskip}
\newenvironment{CSLReferences}[2] % #1 hanging-ident, #2 entry spacing
 {% don't indent paragraphs
  \setlength{\parindent}{0pt}
  % turn on hanging indent if param 1 is 1
  \ifodd #1
  \let\oldpar\par
  \def\par{\hangindent=\cslhangindent\oldpar}
  \fi
  % set entry spacing
  \setlength{\parskip}{#2\cslentryspacingunit}
 }%
 {}
\usepackage{calc}
\newcommand{\CSLBlock}[1]{#1\hfill\break}
\newcommand{\CSLLeftMargin}[1]{\parbox[t]{\csllabelwidth}{#1}}
\newcommand{\CSLRightInline}[1]{\parbox[t]{\linewidth - \csllabelwidth}{#1}\break}
\newcommand{\CSLIndent}[1]{\hspace{\cslhangindent}#1}

\usepackage{orcidlink,thumbpdf,lmodern}

\newcommand{\class}[1]{`\code{#1}'}
\newcommand{\fct}[1]{\code{#1()}}
\makeatletter
\@ifpackageloaded{tcolorbox}{}{\usepackage[many]{tcolorbox}}
\@ifpackageloaded{fontawesome5}{}{\usepackage{fontawesome5}}
\definecolor{quarto-callout-color}{HTML}{909090}
\definecolor{quarto-callout-note-color}{HTML}{0758E5}
\definecolor{quarto-callout-important-color}{HTML}{CC1914}
\definecolor{quarto-callout-warning-color}{HTML}{EB9113}
\definecolor{quarto-callout-tip-color}{HTML}{00A047}
\definecolor{quarto-callout-caution-color}{HTML}{FC5300}
\definecolor{quarto-callout-color-frame}{HTML}{acacac}
\definecolor{quarto-callout-note-color-frame}{HTML}{4582ec}
\definecolor{quarto-callout-important-color-frame}{HTML}{d9534f}
\definecolor{quarto-callout-warning-color-frame}{HTML}{f0ad4e}
\definecolor{quarto-callout-tip-color-frame}{HTML}{02b875}
\definecolor{quarto-callout-caution-color-frame}{HTML}{fd7e14}
\makeatother
\makeatletter
\makeatother
\makeatletter
\makeatother
\makeatletter
\@ifpackageloaded{caption}{}{\usepackage{caption}}
\AtBeginDocument{%
\ifdefined\contentsname
  \renewcommand*\contentsname{Table of contents}
\else
  \newcommand\contentsname{Table of contents}
\fi
\ifdefined\listfigurename
  \renewcommand*\listfigurename{List of Figures}
\else
  \newcommand\listfigurename{List of Figures}
\fi
\ifdefined\listtablename
  \renewcommand*\listtablename{List of Tables}
\else
  \newcommand\listtablename{List of Tables}
\fi
\ifdefined\figurename
  \renewcommand*\figurename{Figure}
\else
  \newcommand\figurename{Figure}
\fi
\ifdefined\tablename
  \renewcommand*\tablename{Table}
\else
  \newcommand\tablename{Table}
\fi
}
\@ifpackageloaded{float}{}{\usepackage{float}}
\floatstyle{ruled}
\@ifundefined{c@chapter}{\newfloat{codelisting}{h}{lop}}{\newfloat{codelisting}{h}{lop}[chapter]}
\floatname{codelisting}{Listing}
\newcommand*\listoflistings{\listof{codelisting}{List of Listings}}
\makeatother
\makeatletter
\@ifpackageloaded{caption}{}{\usepackage{caption}}
\@ifpackageloaded{subcaption}{}{\usepackage{subcaption}}
\makeatother
\makeatletter
\@ifpackageloaded{tcolorbox}{}{\usepackage[many]{tcolorbox}}
\makeatother
\makeatletter
\@ifundefined{shadecolor}{\definecolor{shadecolor}{rgb}{.97, .97, .97}}
\makeatother
\makeatletter
\makeatother
\ifLuaTeX
  \usepackage{selnolig}  % disable illegal ligatures
\fi
\IfFileExists{bookmark.sty}{\usepackage{bookmark}}{\usepackage{hyperref}}
\IfFileExists{xurl.sty}{\usepackage{xurl}}{} % add URL line breaks if available
\urlstyle{same} % disable monospaced font for URLs
\hypersetup{
  pdftitle={Imputation of Incomplete Multilevel Data with R},
  pdfauthor={Hanne I. Oberman; Johanna Muñoz; Valentijn M.T. de Jong; Gerko Vink; Thomas P.A. Debray},
  pdfkeywords={missing data, multilevel, clustering, mice, R},
  colorlinks=true,
  linkcolor={blue},
  filecolor={Maroon},
  citecolor={Blue},
  urlcolor={Blue},
  pdfcreator={LaTeX via pandoc}}

%% -- Article metainformation (author, title, ...) -----------------------------

%% Author information
\author{Hanne I. Oberman~\orcidlink{0000-0003-3276-2141}\\Utrecht
University \And Johanna
Muñoz~\orcidlink{0000-0002-2384-5415}\\University Medical Center
Utrecht \AND Valentijn M.T. de
Jong~\orcidlink{0000-0001-9921-3468}\\University Medical Center
Utrecht \And Gerko Vink~\orcidlink{0000-0001-9767-1924}\\University
Medical Center Utrecht \AND Thomas P.A.
Debray~\orcidlink{0000-0002-1790-2719}\\University Medical Center
Utrecht}
\Plainauthor{Hanne I. Oberman, Johanna Muñoz, Valentijn M.T. de
Jong, Gerko Vink, Thomas P.A. Debray} %% comma-separated

\title{Imputation of Incomplete Multilevel Data with R}
\Plaintitle{Imputation of Incomplete Multilevel Data with
R} %% without formatting

%% an abstract and keywords
\Abstract{This tutorial illustrates the imputation of incomplete
multilevel data with the \proglang{R} packackage \pkg{mice}. Our scope
is only simple multilevel models, to show how imputation can yield less
biased estimates from incomplete clustered data. More complex models can
be accomodated, but are outside the scope of this paper. Incomplete
multilevel data requires careful consideration of the missing data
problem and analysis strategy. In this tutorial, we focus on a popular
strategy for accommodating missingness in multilevel data: replacing the
missing data with one or more plausible values, i.e.,
imputation.Imputation separates the missing data problem from the main
analysis and the completed data can be analyzed as if it has been fully
observed. This tutorial illustrates the imputation of incomplete
multilevel data with the statistical programming language R. We aim to
show how imputation can yield less biased estimates from incomplete
clustered data. We provide practical guidelines and code snippets for
different missing data situations, including non-ignorable missingness
mechanisms. For brevity, we focus on multilevel imputation using chained
equations with the R mice package and its adjacent packages.}

%% at least one keyword must be supplied
\Keywords{missing
data, multilevel, clustering, \pkg{mice}, \proglang{R}}
\Plainkeywords{missing data, multilevel, clustering, mice, R}

%% publication information
%% NOTE: Typically, this can be left commented and will be filled out by the technical editor
%% \Volume{50}
%% \Issue{9}
%% \Month{June}
%% \Year{2012}
%% \Submitdate{2012-06-04}
%% \Acceptdate{2012-06-04}
%% \setcounter{page}{1}
%% \Pages{1--xx}

%% The address of (at least) one author should be given
%% in the following format:
\Address{
Hanne I. Oberman\\
Methodology and Statistics\\
Padualaan 14\\
Utrecht The Netherlands\\
E-mail: \email{h.i.oberman@uu.nl}\\
URL: \url{https://www.hanneoberman.github.io}\\
\\~
Johanna Muñoz\\
Julius Center for Health Sciences and Primary Care\\
Universiteitsweg 100\\
Utrecht The Netherlands\\
\\~
Valentijn M.T. de Jong\\
Julius Center for Health Sciences and Primary Care\\
Utrecht The Netherlands\\
\\~
Gerko Vink\\
Julius Center for Health Sciences and Primary Care\\
Universiteitsweg 100\\
Utrecht The Netherlands\\
\\~
Thomas P.A. Debray\\
Julius Center for Health Sciences and Primary Care\\
Universiteitsweg 100\\
Utrecht The Netherlands\\
\\~

}

\begin{document}
\maketitle
\ifdefined\Shaded\renewenvironment{Shaded}{\begin{tcolorbox}[boxrule=0pt, sharp corners, frame hidden, borderline west={3pt}{0pt}{shadecolor}, interior hidden, enhanced, breakable]}{\end{tcolorbox}}\fi

\hypertarget{sec-intro}{%
\section{Introduction: Clustering and incomplete data}\label{sec-intro}}

\begin{enumerate}
\def\labelenumi{\arabic{enumi}.}
\tightlist
\item
  missing data occur often in data with human subjects
\item
  missing data may be resolved, but need to be handled in accordance
  with the analysis of scientific interest
\item
  in human-subjects research, there is often clustering, which may be
  captured with multilevel modeling techniques
\item
  if the analysis of scientific interest is a multilevel model, the
  missing data handling method should accommodate the multilevel
  structure of the data
\item
  both missingness and multilevel structures require advanced
  statistical techniques
\item
  this tutorial sets out to facilitate empirical researchers in
  accommodating both multilevel structures as well as missing data.
\item
  we illustrate the use of the software by means of three case studies
  from the social and biomedical sciences.
\end{enumerate}

In hierarchical datasets, clustering is a concern because the
homoscedasticity in the error terms cannot be assumed across clusters
and the relationship among variables may vary at different hierarchical
levels. When multiple imputation is used to deal with missing data, as
the imputation and analysis process is performed separately, it is
necessary that imputation model being congenial with the main analysis
model (Meng, 1994), e.g.~if the main model accounts for the hierarchical
structure also imputation model should do it (Audigier, 2021). Not
including clustering into the imputation process may lead to effect
estimates with smaller standard errors and inflated type I error.

\hypertarget{overview-of-software}{%
\subsection{overview of software}\label{overview-of-software}}

The popular \pkg{mice} package in \proglang{R} \citet{R} TODO:
\citet{Hanne}(Is this section relevant?)

\hypertarget{scope}{%
\subsection{Scope}\label{scope}}

This papers serves as a tutorial for imputing incomplete multilevel data
with \pkg{mice} in \proglang{R}. \pkg{mice} has become the de-facto
standard for imputation by chained equations, which iteratively solves
the missingness on a variable-by-variable basis. \pkg{mice} is known to
yield valid inferences under many different missing data circumstances
\citep{buur18}.

We provide practical guidelines and code snippets for different missing
data situations, including non-ignorable mechanisms. For reasons of
brevity, we focus on multilevel imputation by chained equations with
\pkg{mice} exclusively; other imputation methods and packages \citep[see
e.g.][ and \citet{grun18}]{audi18} are outside the scope of this
tutorial. Assumed knowledge includes basic familiarity with the
\pkg{lme4} notation for multilevel models (see Table \ref{tab:mod}).

We illustrate imputation of incomplete multilevel data using three case
studies:

\begin{itemize}
\tightlist
\item
  \texttt{popmis} from the \pkg{mice} package \citep[simulated data on
  perceived popularity, \(n = 2,000\) pupils across \(N = 100\) schools
  with data that are MAR,][]{mice};
\item
  \texttt{impact} from the \pkg{metamisc} package \citep[empirical data
  on traumatic brain injuries, \(n = 11,022\) patients across \(N = 15\)
  studies with data that are MAR,][]{metamisc};
\item
  \texttt{obesity} from the \pkg{micemd} package {[}simulated data on
  obesity, \(n = 2,111\) patients across \(N = 5\) regions with data
  that are MNAR{]}.
\end{itemize}

For each of these datasets, we discuss the nature of the missingness,
choose one or more imputation models and evaluate the imputed data, but
we will also highlight one specific aspect of the imputation workflow.

This tutorial is dedicated to readers who are unfamiliar with multiple
imputation. More experienced readers can skip the background section and
introduction (case study 1) and directly head to practical applications
of multilevel imputation under MAR conditions (case study 2,
Section~\ref{sec-impact}) or under MNAR conditions (case study 3).

\hypertarget{sec-models}{%
\section{Background}\label{sec-models}}

\hypertarget{concepts-in-multilevel-data}{%
\subsection{Concepts in multilevel
data}\label{concepts-in-multilevel-data}}

Many datasets include individuals that are clustered together, for
example in geographic regions, or even different studies. In the
simplest case, individuals (e.g., students) are nested within a single
cluster (e.g., school classes). More complex clustered structures may
occur when there are multiple hierarchical levels (e.g., students in
different schools or patients within hospitals within regions across
countries), or when the clustering is non-nested (e.g., electronic
health record data from diverse settings and populations within large
databases). With clustered data we generally assume that individuals
from the same cluster tend to be more similar than individuals from
other clusters. In statistical terms, this implies that observations
from the same cluster are not independent and may in fact be correlated.
If this correlation is left unaddressed, estimates of \emph{p} values,
confidence intervals even model parameters are prone to bias
\citep{loca01}. Statistical methods for clustered data typically adopt
hierarchical models that explicitly describe the grouping of
observations. These models are also known as `multilevel models',
`hierarchical models', `mixed effect models', `random effect models',
and in the context of time-to-event data as `frailty models'. Table
\ref{tab:clus0} provides an overview of some key concepts in multilevel
modeling.

\begin{table}

\caption{Concepts in multilevel methods}
\centering
\begin{tabular}[t]{>{\raggedright\arraybackslash}p{3cm}>{\raggedright\arraybackslash}p{12cm}}
\toprule
Concept & Details\\
\midrule
Sample units & Units of the population from which measurements are taken in a sample, e.g., students.\\
Cluster & Variable that specify the cluster or agruppation, e.g., Classroom\\
Hierarchical data & Data are grouped into clusters at different levels, observations belonging to the same cluster are expected to share certain characteristics.\\
Level-1 & Variable that varies within a cluster, eg. Test score\\
Level-2 & Variable that does not vary within a cluster but between, e.g. teacher experience.\\
\addlinespace
Hierarchical model & Model accounting for dependant observations relying on certain parameters ( within cluster) which in turn depend on other parameters (between cluster)\\
Fixed effect & Effects that are constant across all sample units, e.g. something that researchers control for and can repeat, such as  a teaching strategy (tutoring after class)\\
Random effect & Effects that are a source of random variation in the data, and whose levels are not fully sampled. e.g. test score tendency during academic year between students due to no controlled factors such as  genetic,family history\\
Mixed effect & Includes fixed and random effects, e.g. the fixed effect would be the treatment effect of a drug and the random effect would be the ID of the hospital where the patient is treated. Multilevel models typically accommodate for variability by including a separate group mean for each cluster e.g random intercept on hospitals. In addition to random intercepts, multilevel models can also include random coefficients and heterogeneous residual error variances across clusters (see e.g. @gelm06, @hox17 and @jong21).\\
ICC & The variability due to clustering is often measured by means of the intraclass coefficient (ICC). The ICC can be seen as the percentage
of variance that can be attributed to the cluster-level, where a high ICC would indicate that a lot of variability is due to the cluster structure.\\
\addlinespace
Stratified intercept & \\
\bottomrule
\end{tabular}
\end{table}

In \proglang{R}, multilevel models may be fitted using package such as
\pkg{lme4} or \pkg{nlme}. For linear mixed-effects models, the function

\begin{verbatim}
lmer(formula, data, ...)
\end{verbatim}

\hypertarget{concepts-in-missing-data}{%
\subsection{Concepts in missing data}\label{concepts-in-missing-data}}

missing data mechanisms etc.

As with any other dataset, clustered datasets may be impacted by
missingness in much the same way. Several strategies can be used to
handle missing data, including complete case analysis and imputation. We
focus on the latter approach and discuss statistical methods for
replacing the missing data with one or more plausible values. Imputation
separates the missing data problem from the analysis and the completed
data can be analyzed as if it were completely observed. It is generally
recommended to impute the missing values more than once to preserve
uncertainty due to missingness and to allow for valid inferences (c.f.
Rubin 1976).

With incomplete clustered datasets we can distinguish between two types
of missing data: sporadic missingness and systematic missingness
\citep{resc13}. Sporadic missingness arises when variables are missing
for some but not all of the units in a cluster \citep{buur18, jola18}.
For example, it is possible that test results are missing for several
students in one or more classes. When all observations are missing
within one or more clusters, data are said to be systematically missing.
Sporadic missingness is visualized in Figure XYZ.

\begin{figure}[h]

{\centering \includegraphics{manuscript_files/figure-pdf/unnamed-chunk-2-1.pdf}

}

\end{figure}

Column \(X_1\) in Figure 1 is completely observed, column \(X_2\) is
systematically missing in cluster 2, and column \(X_3\) is sporadically
missing. To analyze these incomplete data, we have to take the nature of
the missingness and the cluster structure into account. For example, the
sporadic missingness in \(X_3\) could be easily amended if this would be
a cluster-level variable (and thus constant within clusters). We could
then just extrapolate the true (but missing) value of \(X_3\) for unit 1
from unit 2, and the value for unit 4 from unit 3. If \(X_3\) would
instead be a unit-level variable (which may vary within clusters), we
could not just recover the unobserved `truth', but would need to use
some kind of missing data method, or discard the incomplete units
altogether (i.e., complete case analysis). Complete case analysis can
however introduce bias in statistical inferences and lowers statistical
power. Further, with the systematic missingness in \(X_2\), it would be
impossible to fit a multilevel model without accommodating the
missingness in some way. Complete case analysis in that case would mean
excluding the entire cluster from the analyses. The wrong choice of
missing data handling method can thus be extremely harmful to the
inferences.

Imputation of missing data requires consideration of the mechanism
behind the missingness. Rubin proposed to distinguish between data that
are missing completely at random (MCAR), data that are missing at random
(MAR) and data that are missing not at random (MNAR; see Table
\ref{tab:miss}). For each of these three missingness generating
mechanisms, different imputation strategies are warranted
(\citet{yuce08} and \citet{hox15}). We here consider the general case
that data are MAR, and expand on certain MNAR situations.

Table 2: Concepts in missing data methods

\begin{longtable}[]{@{}
  >{\raggedright\arraybackslash}p{(\columnwidth - 2\tabcolsep) * \real{0.1739}}
  >{\raggedright\arraybackslash}p{(\columnwidth - 2\tabcolsep) * \real{0.8261}}@{}}
\toprule()
\begin{minipage}[b]{\linewidth}\raggedright
\textbf{Concept}
\end{minipage} & \begin{minipage}[b]{\linewidth}\raggedright
\textbf{Details}
\end{minipage} \\
\midrule()
\endhead
MCAR & Missing Completely At Random, where the probability to be missing
is equal \\
& across all data entries \\
MAR & Missing At Random, where the probability to be missing depends on
observed \\
& information \\
MNAR & Missing Not At Random (MNAR), where the probability to be
missing \\
& depends on unrecorded information, making the missingness
non-ignorable \\
& \citep{rubi76, meng94}. \\
& \\
\bottomrule()
\end{longtable}

\hypertarget{imputation-with-mice}{%
\subsection{Imputation with mice}\label{imputation-with-mice}}

The \proglang{R} package \pkg{mice} provides a framework for imputing
incomplete data on a variable-by-variable basis. The \fct{mice} function
allows users to flexibly specify how many times and under what model the
missing data should be imputed. This is reflected in the first four
function arguments

\begin{verbatim}
mice(data, m, method, predictorMatrix, ...)
\end{verbatim}

where \texttt{data} refers to the incomplete dataset, \texttt{m}
determines the number of imputations, \texttt{method} denotes the
functional form of the imputation model and \texttt{predictorMatrix}
specifies the interrelational dependencies between variables and
imputation models (i.e., the set of predictors to be used for imputing
each incomplete variable).

\begin{tcolorbox}[enhanced jigsaw, opacityback=0, breakable, bottomrule=.15mm, colback=white, left=2mm, colframe=quarto-callout-color-frame, rightrule=.15mm, leftrule=.75mm, arc=.35mm, toprule=.15mm]

Box 2. The \texttt{methods}.

\end{tcolorbox}

\begin{tcolorbox}[enhanced jigsaw, opacityback=0, breakable, bottomrule=.15mm, colback=white, left=2mm, colframe=quarto-callout-color-frame, rightrule=.15mm, leftrule=.75mm, arc=.35mm, toprule=.15mm]

TODO: @ Hanne i added a wide explanation about hte prediction matrix in
the Model Specification subsection, so you can check it and consider to
modify your box.. because i dont think is usefull to repeat information
across the paper\ldots{}

Box 3. The predictor matrix. The entries corresponding to the level-1
predictors are coded with a 3, indicating that both the original values
as well as the cluster means of the predictor are included into the
imputation model. The entry of 4 in the predictor matrix adds three
variables to the imputation model for the imputation model predictor:
the value of the predictor, the cluster means of the predictor and the
random slopes of the predictor. - -2 = cluster variable - 1 = overall
effect - 3 = overall + group-level effect - 4 = individual-level
(random) and group-level (fixed) effect

\end{tcolorbox}

\hypertarget{sec-workflow}{%
\section{Multilevel imputation workflow}\label{sec-workflow}}

There are different strategies that can be adopted in the imputation
process that account for clustering: inclusion of cluster indicator
variable, performing a separate imputation process for each cluster, or
performing a simultaneous imputation process by using an imputation
method that accounts for clustering.\cite{zotero-516}

The selection of each strategy depends mainly on the assumptions in the
main analysis and also on the restriction of the analyzed data.

Regarding the restrictions imposed by the nature of the data, i.e., the
sample size, the use of cluster indicator variables is restricted in
datasets where there are not many clusters and many observations per
cluster (Graham, 2009). The last restriction is also required when
imputations are performed on each cluster separately. When this
restriction cannot be achieved, one can use an imputation model that
simultaneously imputes all clusters using a hierarchical model (Allison
2002).

Under this hierarchical imputation model, observations within clusters
are correlated and this correlation is modelled by a random effect so
the hierarchical model can be estimated even when there are few
observations per cluster. However, this strategy is best suited for
balanced data (Grund, 2017) and when the number of clusters is adequate
sufficient for the random effects model (Austin,2018).

Here it is important to evaluate the assumptions imposed by the main
model, for instance the cluster indicator strategy bias the estimator
when the model is based on a hierarchical model (Taaljard,2008).
Although an imputation strategy congenial with the main model is
preferred, it is important to consider whether a complex imputation
model is appropriate for the data, as a less complex imputation
strategies may also lead to unbiased estimates in certain scenarios
(Bailey 2020). Below we provide a imputation workflow that can be used
in general to impute cluster data.

\hypertarget{exploration-of-available-data}{%
\subsection{Exploration of Available
Data}\label{exploration-of-available-data}}

First, explore the data, preferably using visual tools. This will help
you choose an imputation model that suits your data. In addition, the
intraclass correlation coefficient (ICC) can be examined to assess
cluster differences, aiding in the choice between methods for dependent
(such as clustering) or independent data.

Next, explore the missingness. Look at the
\textbf{proportion of missing values} in the dataset variables.
Variables for which most data is missing may contribute little in the
analysis, and may be removed from the analysis model. Removing them also
from the imputation model can lower the risk of multicollinearity or
computational issues, especially with certain parametric imputation
methods. You can also identify predictors for the imputation model by
using inflow criteria to see connections between missing data in one
variable and observed variables, and outflow criteria to identify
connections between observed values in one variable and missing data in
others.

Check the missing patterns at cluster level, this can help you to select
the most appropriate imputation approach in terms of computational
efficiency (e.g., simpler regression imputation versus FCS in univariate
patterns).

\hypertarget{assess-estimation-procedure-robustness-to-missing-data}{%
\subsection{Assess Estimation Procedure Robustness to Missing
Data}\label{assess-estimation-procedure-robustness-to-missing-data}}

Before diving into imputation, make sure your main model can handle
missing data. Sometimes, simpler methods like complete case analysis
might be suffice, especially if your missingness is low (usually
\textless5\%). There are scenarios in which specific Maximum Likelihood
(ML) estimation methods outperform Multiple Imputation (MI) methods, for
instance when the response variable is the sole incomplete variable,
mixed models demonstrate robustness to missing data under the Missing at
Random (MAR) assumption and with a correct variance-covariance
specification.\cite{molenberghs2007}.

\hypertarget{pre-imputation}{%
\subsection{Pre-imputation}\label{pre-imputation}}

Clean up your dataset before the imputation process. For imputation,
include the variables necessary for your analysis, depending on your
analysis this may be the exposure or intervention, outcome, covariates,
confounders, stratification factors, and the cluster variable. Think
about adding other useful variables, even if they're not in the main
model, such as instrumental or auxiliary variables which might improve
the accuracy of the imputation model, especially if they're associated
to the probability of missingness for some incomplete variables.

Figure out if you can directly impute incomplete variables by using
deductive imputation. This involves inferring missing values based on
logical connections between variables. It's especially useful for
variables that depend on each other, like calculating BMI from weight
and height.

Deductive imputation is also useful for getting values for level-1
variables from level-2 ones, like in Individual Participant Data (IPD).
For example, deductive imputation can be employed to infer missing test
values for patients associated with a center unit that lacks a test
procedure or machine, utilizing accessible metadata or protocol details.

\hypertarget{setting-imputation-model}{%
\subsection{Setting Imputation Model}\label{setting-imputation-model}}

\hypertarget{clustering-inclusion}{%
\paragraph{Clustering Inclusion}\label{clustering-inclusion}}

When it comes to handling clusters during the imputation process, you've
got a few options. You can use a cluster indicator variable, run
separate imputation for each cluster, or go for a simultaneous
imputation method that takes clustering into account \cite{eddings}.

Which strategy you choose depends on the assumptions in your main
analysis and the limitations of your data. If your analysis do not use a
hierarchical model (like a descriptive approach) and you have a small
number of clusters with lots of observations in each, using a cluster
indicator or separate imputation might be the way to go
\cite{graham2009}. Conversely, if you have more clusters or fewer
observations per cluster, you might want to try a simultaneous
hierarchical imputation model \cite{Allison_2002}.

In a hierarchical imputation model, random intercepts and random effects
model the correlations between observations within clusters, making it
possible to estimate even with a small number of observations per
cluster. There are various proposed multiple imputation models based on
hierarchical models, each with its own set of assumptions
\cite{audigier}.

If your analysis uses a hierarchical model, make sure the assumptions of
your imputation model match up. For example, using a cluster indicator
approach may lead to bias estimates if your model is based on a
hierarchical structure \cite{taljaard2008,speidel2018}. Even if you
prefer an imputation strategy that aligns with your main model, check if
it suits your data; sometimes simpler strategies can give unbiased
estimates in certain scenarios \cite{bailey2020}.

\hypertarget{choice-of-individual-imputation-methods}{%
\subsubsection{Choice of Individual Imputation
Methods}\label{choice-of-individual-imputation-methods}}

Start by choosing the imputation model for each incomplete variable in
your dataset. The mice package suggests methods based on variable types
for non-clustered variables, and for the clustered ones you can use the
micemd package's find.defaultMethod() function. This function selects
from different 2l imputation methods based on cluster size and the
proportion of missing data in each cluster.

Besides the package-defined imputation methods, you can specify custom
methods using the ``I formula''. This lets you calculate deterministic
variables during the imputation or tweak imputation methods based on
specific conditions, like conditioning the imputation model to the level
of an incomplete covariate (e.g., a pregnancy test for females).

\hypertarget{model-specification}{%
\subsubsection{Model Specification}\label{model-specification}}

The imputation model must be congenial with the main model
\cite{meng1994}. Congeniality issues arise when the imputation model and
the main model make different assumptions, often due to the omission of
a polynomial or interaction term or the use of transformed variables.

The imputation model can incorporate additional terms compared to the
main model without causing compatibility issues. For instance, it's
recommended to include the outcome variable in the imputation model for
prediction variables \cite{moons2006a}. In cases where the outcome is
time-to-event, the Nelson-Aalen estimate of the time to the event should
be added as a covariate in the imputation model, as well as the event
indicator {[}REF: White IR, Royston P. Imputing missing covariate values
for the Cox model. Stat Med. 2009 Jul 10;28(15):1982-98. Available from:
doi.org/10.1002/sim.3618{]}. Additionally, including auxiliary
variables, even if not part of the main model, can be linked to the
probability of missingness, improving the likelihood of meeting the
Missing at Random (MAR) assumption and enhancing estimation efficiency
\cite{hardt2012a}.

Imputation models are specified on a variable basis, either using a
prediction matrix in the pred parameter or through a list of formulas in
the formula parameter. In the prediction matrix option, the type of each
predictor variable is specified for each incomplete variable (see
table).

\begin{align}
y_{ij} =& (\beta_0 + b_{0_i})+ \beta_1x_{1_{ij}}+\beta_2x_{2_{i.}} \nonumber \\
+& (\beta_3 +  b_{3_i})x_{3_{ij}} \nonumber \\
+& \beta_4x_{4_{ij}} + \beta_{m4}\overline{x_{4_{i.}}}\nonumber \\
+& (\beta_5 +  b_{5_i})x_{5_{ij}} +\beta_{m5}\overline{x_{5_{i.}}}\nonumber \\
\end{align}

\begin{tabular}{ll}
     \toprule
Type & Definition \\
  \midrule
 1  & Fixed variable, e.g., level-1  $x_{1_{ij}}$ or  level-2 $x_{2_{i.}}$ \\
 2  & Random variable, e.g., $x_{3_{ij}}$\\
 3  & *Fixed variable with cluster mean $x_{4_{ij}}$\\
 4  & *Random variable with cluster mean $x_{5_{ij}}$\\
  \midrule
-2  & Cluster variable, in this case the one defined by j index  \\
-3. & Random variable only included on selection model (Heckman model)\\
-4. & Random variable only included on main model (Heckman model)\\
      \bottomrule
\end{tabular}

\begin{itemize}
\tightlist
\item
  It has been advised to use variables of type 3 and 4, as the inclusion
  of the means of the cluster is beneficial on FCS \cite{mistler2017}.
\end{itemize}

Recipes have been proposed for imputing incomplete level-1 and level-2
variables for hierarchical models \cite{buuren2018a} (\(\S\) 7.10),
which can be convenient to follow when dealing with numerous variables
(interaction terms at different levels) and models with many random
effects that are prone to convergency problems due to overspecification
in the imputation model. These rules were designed to ensure
compatibility among the conditionally specified imputation models and
congeniality between the imputation and the main model. However, they
may not be applicable to all 2l imputation methods. For instance, the
2l.2stage imputation methods of the micemd package only allow the
inclusion of random predictor variables (2).

On the other hand, the formula option is useful in specifying complex
imputation models with polynomial terms or interactions and compared
with the prediction matrix method do not requires the inclusion of
additional terms as Just Another Variable (JAV) \cite{buur18}(\(\S\)
6.4).

For some interaction terms, including treatment interaction effects, it
has been suggested to conduct separate imputation by treatment group
\cite{zhang2023}. Additionally, one might apply random forest or deep
learning methods, as these can handle interaction and non-linear terms
without requiring the explicit specification of an imputation model.

\hypertarget{post-imputation}{%
\subsection{Post-Imputation}\label{post-imputation}}

During the imputation process, certain issues may arise that halt the
process. In hierarchical model imputations, many issues are related to
overfitting the imputation model. To troubleshoot, it is recommended to
inspect the imputation log file for variables causing problems. One
approach to address this is to reduce the number of predictors, by
applying the previous referred recepies step by step or by using
functions like quickpred. Another option is to consider variable
transformations, such as scaling when the model is invariant to linear
transformations e.g., random intercept models. Also, adjusting the level
of the hierarchical model (e.g., using a homogeneous variance imputation
method or a 1l model) can also be beneficial.

It is crucial to check the range of imputed variables, as excessively
large imputed values for one predictor may trigger convergence issues in
other variables. To tackle this, including post-processing
specifications on problematic variables or utilizing imputation models
like Predicted Mean Matching (PMM) can ensure that imputed values align
with observable values.

In some situations, adopting a separate imputation strategy might be
worth considering. For example, in analyses involving multiple
endpoints, conducting distinct imputation processes for each endpoint
could be more effective than a unified imputation approach.

\hypertarget{convergence-and-sensitivity-analysis}{%
\subsection{Convergence and Sensitivity
Analysis}\label{convergence-and-sensitivity-analysis}}

Before starting the analysis of each imputed dataset, it is crucial to
validate the convergence of the imputation process. This is commonly
accomplished through trace plots that depict the mean and variance of
the incomplete variables across iterations. These plots serve to uncover
potential circular issues or the need for additional iterations.
Additionally, it is also important to verify that imputed values fall
within a plausible range and also to check the distribution of imputed
variables, ensuring that the imputed variable distribution aligns with
the distribution of observed values (under the MAR assumption). An
alternative approach involves assessing the prediction accuracy of the
imputation method \cite{cai2023}.

While the majority of Multiple Imputation by Chained Equations (MICE)
methods are based on Missing at Random (MAR) assumptions, field expert
input may suggest that the Missing Not at Random (MNAR) mechanism could
be plausible for certain variables. An MNAR variable is one in which the
probability of missingness depends on an unobservable variable. This can
occur when missingness is associated with the incomplete value itself
(self-marking) or when there is an unobserved variable linked to both
the value and the probability of missingness of the incomplete value
(indirectly non-informative). Specifically, for the indirectly
non-informative case in hierarchical datasets, imputation methods based
on the Heckman method can be
considered.\cite{hammon2020,hammon2022,munoz2023b}

\begin{center}\rule{0.5\linewidth}{0.5pt}\end{center}

\hypertarget{sec-illustrations}{%
\section{Illustrations}\label{sec-illustrations}}

In this section, we demonstrate the workflow using three case studies.

\hypertarget{setup}{%
\subsection{Setup}\label{setup}}

\begin{verbatim}
R> set.seed(123)         # for reproducibility
R> library(lme4)         # for multilevel modeling
R> library(nlme)         # for multilevel modeling
R> library(mice)         # for imputation
R> library(miceadds)     # for multilevel imputation methods
R> library(micemd)       # for selection-model imputation methods
R> library(mitml)        # for multilevel parameter pooling
R> library(broom)        # for clean model estimates
R> library(broom.mixed)  # for multilevel model estimates
R> library(dplyr)        # for data wrangling
R> library(ggmice)       # for visualization
R> library(ggplot2)      # for visualization
\end{verbatim}

\hypertarget{popularity-data}{%
\subsection{Popularity data}\label{popularity-data}}

In this section we will go over the different steps involved with
imputing incomplete multilevel data with the R package mice. We consider
the simulated \texttt{popmis} dataset, which included pupils
(\(n = 2000\)) clustered within schools (\(N = 100\)). The following
variables are of primary interest:

\begin{itemize}
\tightlist
\item
  \texttt{school}, school identification number (clustering variable);
\item
  \texttt{popular}, pupil popularity (self-rating between 0 and 10;
  unit-level);
\item
  \texttt{sex}, pupil sex (0 = boy, 1 = girl; unit-level);
\item
  \texttt{texp}, teacher experience (in years; cluster-level).
\end{itemize}

The analysis model corresponding to this dataset is multilevel
regression with random intercepts for the different schools. We will
estimate the association between the pupils' sex and their popularity
score. This model can be expressed in \texttt{lme4} code as:

\begin{verbatim}
popular ~ 1 + sex + (1 | school)
\end{verbatim}

Given the \(j\)-th student belonging to the \(i\)-th school, the main
model can be formulated as:

\[ popular_{ij} = (\beta_0 + b_i) + \beta_1sex_{ij}+ \beta_2texp_{ij}+\epsilon_{ij}\]
where \(\epsilon_{ij}\) corresponds to the error term.

We load the data into the environment with

\begin{verbatim}
R> data("popmis", package = "mice")
\end{verbatim}

and select the relevant variables

\begin{verbatim}
R> dat <- popmis[, c("school", "popular", "texp", "sex")] 
\end{verbatim}

which results in the following data structure.

\begin{verbatim}
R> head(dat)
\end{verbatim}

\begin{verbatim}
  school popular texp sex
1      1      NA   24   1
2      1      NA   24   0
3      1       7   24   1
4      1      NA   24   1
5      1      NA   24   1
6      1       7   24   0
\end{verbatim}

The association of interest can be visualized with \texttt{ggmice},

\begin{verbatim}
ggmice(dat, aes(sex, popular)) +
  geom_jitter()
\end{verbatim}

\begin{figure}[h]

{\centering \includegraphics{manuscript_files/figure-pdf/unnamed-chunk-7-1.pdf}

}

\caption{Scatterplot of student popularity by sex}

\end{figure}

where missing datapoints in the \texttt{popular} variable are
represented by red points on the X-axis of the figure.

With the \texttt{ggmice} function \texttt{plot\_pattern} we can
visualize the missing data pattern

\begin{verbatim}
R> plot_pattern(dat)
\end{verbatim}

\begin{figure}[h]

{\centering \includegraphics{manuscript_files/figure-pdf/fig-pattern-1.pdf}

}

\caption{\label{fig-pattern}Missing data pattern.}

\end{figure}

which shows us that the missingness is univariate and sporadic.

To develop the best imputation model for the incomplete variable
\texttt{popular}, we need to know whether the observed values of
\texttt{popular} are related to observed values of other variables. Plot
the pair-wise complete correlations in the incomplete data

\begin{verbatim}
R> plot_corr(dat)
\end{verbatim}

\begin{figure}[h]

{\centering \includegraphics{manuscript_files/figure-pdf/unnamed-chunk-9-1.pdf}

}

\caption{Pair-wise correlations.}

\end{figure}

This shows us that not just the analysis-model variable \texttt{sex},
but also the cluster-level covariate teacher experience, \texttt{texp},
may be a useful as an imputation model predictor. Moreover, the
missingness in \texttt{popular} may depend on the observed values of
other variables. With \texttt{ggmice()} we can visualize the
distribution of the teacher experience for cases where \texttt{popular}
is observed and cases where \texttt{popular} is missing.

\begin{verbatim}
R> ggmice(dat, aes(texp)) + 
+  geom_density() +
+  facet_wrap(~is.na(popular))
\end{verbatim}

\begin{figure}[h]

{\centering \includegraphics{manuscript_files/figure-pdf/unnamed-chunk-10-1.pdf}

}

\end{figure}

It appears that students with a missing value for \texttt{popular} are
in clusters with a slightly higher \texttt{texp} value.

\begin{verbatim}
t.test(dat$texp ~ is.na(dat$popular)) |> 
  tidy() |>
  kable()
\end{verbatim}

\begin{longtable}[]{@{}
  >{\raggedleft\arraybackslash}p{(\columnwidth - 18\tabcolsep) * \real{0.0924}}
  >{\raggedleft\arraybackslash}p{(\columnwidth - 18\tabcolsep) * \real{0.0840}}
  >{\raggedleft\arraybackslash}p{(\columnwidth - 18\tabcolsep) * \real{0.0840}}
  >{\raggedleft\arraybackslash}p{(\columnwidth - 18\tabcolsep) * \real{0.0924}}
  >{\raggedleft\arraybackslash}p{(\columnwidth - 18\tabcolsep) * \real{0.0840}}
  >{\raggedleft\arraybackslash}p{(\columnwidth - 18\tabcolsep) * \real{0.0840}}
  >{\raggedleft\arraybackslash}p{(\columnwidth - 18\tabcolsep) * \real{0.0924}}
  >{\raggedleft\arraybackslash}p{(\columnwidth - 18\tabcolsep) * \real{0.0840}}
  >{\raggedright\arraybackslash}p{(\columnwidth - 18\tabcolsep) * \real{0.2017}}
  >{\raggedright\arraybackslash}p{(\columnwidth - 18\tabcolsep) * \real{0.1008}}@{}}
\toprule()
\begin{minipage}[b]{\linewidth}\raggedleft
estimate
\end{minipage} & \begin{minipage}[b]{\linewidth}\raggedleft
estimate1
\end{minipage} & \begin{minipage}[b]{\linewidth}\raggedleft
estimate2
\end{minipage} & \begin{minipage}[b]{\linewidth}\raggedleft
statistic
\end{minipage} & \begin{minipage}[b]{\linewidth}\raggedleft
p.value
\end{minipage} & \begin{minipage}[b]{\linewidth}\raggedleft
parameter
\end{minipage} & \begin{minipage}[b]{\linewidth}\raggedleft
conf.low
\end{minipage} & \begin{minipage}[b]{\linewidth}\raggedleft
conf.high
\end{minipage} & \begin{minipage}[b]{\linewidth}\raggedright
method
\end{minipage} & \begin{minipage}[b]{\linewidth}\raggedright
alternative
\end{minipage} \\
\midrule()
\endhead
-0.2599581 & 14.15278 & 14.41274 & -0.8731291 & 0.3827095 & 1796.115 &
-0.8438947 & 0.3239786 & Welch Two Sample t-test & two.sided \\
\bottomrule()
\end{longtable}

Although there are no significant differences in the distribution of
\texttt{texp} depending on the missingness indicator of
\texttt{popular}, this variable can serve as auxiliary variable in the
imputation of \texttt{popular}.

\begin{verbatim}
R> meth <- make.method(dat)
R> meth
\end{verbatim}

\begin{verbatim}
 school popular    texp     sex 
     ""   "pmm"      ""      "" 
\end{verbatim}

\begin{verbatim}
R> pred <- quickpred(dat)
R> pred
\end{verbatim}

\begin{verbatim}
        school popular texp sex
school       0       0    0   0
popular      0       0    1   1
texp         0       0    0   0
sex          0       0    0   0
\end{verbatim}

Adjust the methods vector.

\begin{verbatim}
R> meth["popular"] <- "2l.pmm"
\end{verbatim}

The \texttt{pmm} method is better (more efficient) because it will still
look for donors (maybe outside of cluster) based on predictive distance,
even for very small clusters.

Adjust the predictor matrix.

\begin{verbatim}
R> pred["popular", "school"] <- -2
R> pred["popular", "sex"] <- 2
\end{verbatim}

Visualize the imputation methods and predictors.

\begin{verbatim}
plot_pred(pred, method = meth)
\end{verbatim}

\begin{figure}[h]

{\centering \includegraphics{manuscript_files/figure-pdf/unnamed-chunk-16-1.pdf}

}

\end{figure}

Impute the data.

\begin{verbatim}
R> imp <- mice(
+  data = dat,
+  method = meth,
+  predictorMatrix = pred,
+  printFlag = FALSE
+)
\end{verbatim}

Evaluate the convergence.

\begin{verbatim}
R> plot_trace(imp)
\end{verbatim}

\begin{figure}[h]

{\centering \includegraphics{manuscript_files/figure-pdf/unnamed-chunk-18-1.pdf}

}

\end{figure}

Troubleshoot non-convergence of imputation model. Evaluate the
distribution of imputed values.

\begin{verbatim}
R> ggmice(imp, aes(popular, group = .imp)) + 
+  geom_density() 
\end{verbatim}

\begin{figure}[h]

{\centering \includegraphics{manuscript_files/figure-pdf/unnamed-chunk-19-1.pdf}

}

\end{figure}

Evaluate the distribution of imputed values.

\begin{verbatim}
R> ggmice(imp, aes(.imp, popular)) + 
+  geom_jitter(alpha = 0.05) +
+    geom_boxplot()
\end{verbatim}

\begin{figure}[h]

{\centering \includegraphics{manuscript_files/figure-pdf/unnamed-chunk-20-1.pdf}

}

\end{figure}

\begin{verbatim}
R> ggmice(imp, aes(sex, popular)) +
+  geom_jitter() +
+  facet_wrap(~ .imp)
\end{verbatim}

\begin{figure}[h]

{\centering \includegraphics{manuscript_files/figure-pdf/unnamed-chunk-21-1.pdf}

}

\end{figure}

Analyze the imputed data.

\begin{verbatim}
R> fit <- with(
+  imp,
+  lmer(popular ~ texp + sex + (1 | school))
+)
\end{verbatim}

Pooling the estimates does not provide estimates of the variance
components.

\begin{verbatim}
R> pool(fit)
\end{verbatim}

\begin{verbatim}
Class: mipo    m = 5 
         term m   estimate         ubar            b            t dfcom
1 (Intercept) 5 3.59902513 0.0284760238 3.461284e-03 0.0326295651  1995
2        texp 5 0.09225928 0.0001140591 2.273382e-05 0.0001413397  1995
3         sex 5 0.85187018 0.0009628869 4.946825e-04 0.0015565060  1995
        df       riv    lambda       fmi
1 216.1756 0.1458610 0.1272938 0.1352573
2 100.6502 0.2391794 0.1930144 0.2085857
3  26.9008 0.6164992 0.3813792 0.4227574
\end{verbatim}

Therefore, \texttt{mitml} is used.

\begin{verbatim}
R> est <- testEstimates(as.mitml.result(fit), extra.pars = TRUE)
\end{verbatim}

Display results in table.

\begin{verbatim}
R> est$estimates |> 
+  round(3) |>
+  kable()
\end{verbatim}

\begin{longtable}[]{@{}
  >{\raggedright\arraybackslash}p{(\columnwidth - 14\tabcolsep) * \real{0.1558}}
  >{\raggedleft\arraybackslash}p{(\columnwidth - 14\tabcolsep) * \real{0.1169}}
  >{\raggedleft\arraybackslash}p{(\columnwidth - 14\tabcolsep) * \real{0.1299}}
  >{\raggedleft\arraybackslash}p{(\columnwidth - 14\tabcolsep) * \real{0.1039}}
  >{\raggedleft\arraybackslash}p{(\columnwidth - 14\tabcolsep) * \real{0.1039}}
  >{\raggedleft\arraybackslash}p{(\columnwidth - 14\tabcolsep) * \real{0.2338}}
  >{\raggedleft\arraybackslash}p{(\columnwidth - 14\tabcolsep) * \real{0.0779}}
  >{\raggedleft\arraybackslash}p{(\columnwidth - 14\tabcolsep) * \real{0.0779}}@{}}
\toprule()
\begin{minipage}[b]{\linewidth}\raggedright
\end{minipage} & \begin{minipage}[b]{\linewidth}\raggedleft
Estimate
\end{minipage} & \begin{minipage}[b]{\linewidth}\raggedleft
Std.Error
\end{minipage} & \begin{minipage}[b]{\linewidth}\raggedleft
t.value
\end{minipage} & \begin{minipage}[b]{\linewidth}\raggedleft
df
\end{minipage} & \begin{minipage}[b]{\linewidth}\raggedleft
P(\textgreater\textbar t\textbar)
\end{minipage} & \begin{minipage}[b]{\linewidth}\raggedleft
RIV
\end{minipage} & \begin{minipage}[b]{\linewidth}\raggedleft
FMI
\end{minipage} \\
\midrule()
\endhead
(Intercept) & 3.599 & 0.181 & 19.924 & 246.857 & 0 & 0.146 & 0.134 \\
texp & 0.092 & 0.012 & 7.760 & 107.369 & 0 & 0.239 & 0.208 \\
sex & 0.852 & 0.039 & 21.592 & 27.501 & 0 & 0.616 & 0.422 \\
\bottomrule()
\end{longtable}

\begin{verbatim}
R> est$extra.pars |> 
+  round(3) |>
+  kable()
\end{verbatim}

\begin{longtable}[]{@{}lr@{}}
\toprule()
& Estimate \\
\midrule()
\endhead
Intercept\textasciitilde\textasciitilde Intercept\textbar school &
0.470 \\
Residual\textasciitilde\textasciitilde Residual & 0.462 \\
ICC\textbar school & 0.504 \\
\bottomrule()
\end{longtable}

\hypertarget{sec-impact}{%
\subsection{IMPACT data}\label{sec-impact}}

The second case study is the \texttt{impact} data from the
\pkg{metamisc} package \citep[empirical data on traumatic brain
injuries, \(n = 11,022\) units across \(N = 15\) clusters,][]{metamisc}.

The \texttt{impact} data set contains traumatic brain injury data on
\(n = 11022\) patients clustered in \(N = 15\) studies with the
following 11 variables:

\begin{itemize}
\tightlist
\item
  \texttt{name} Name of the study,
\item
  \texttt{type} Type of study (RCT: randomized controlled trial, OBS:
  observational cohort),
\item
  \texttt{age} Age of the patient,
\item
  \texttt{motor\_score} Glasgow Coma Scale motor score,
\item
  \texttt{pupil} Pupillary reactivity,
\item
  \texttt{ct} Marshall Computerized Tomography classification,
\item
  \texttt{hypox} Hypoxia (0=no, 1=yes),
\item
  \texttt{hypots} Hypotension (0=no, 1=yes),
\item
  \texttt{tsah} Traumatic subarachnoid hemorrhage (0=no, 1=yes),
\item
  \texttt{edh} Epidural hematoma (0=no, 1=yes),
\item
  \texttt{mort} 6-month mortality (0=alive, 1=dead).
\end{itemize}

Check if there is systematic missingness in this dataset. For
illustration purposes, we made Marshall Computerized Tomography
classification (ct) systematically missing.

The analysis model for this dataset is a prediction model with
\texttt{mort} as the outcome. In this tutorial we'll estimate the
adjusted prognostic effect of \texttt{ct} on mortality outcomes. The
estimand is the adjusted odds ratio for \texttt{ct}, after including
\texttt{type}, \texttt{age} \texttt{motor\_score} and \texttt{pupil}
into the analysis model:

\begin{verbatim}
mort ~ type + age + motor_score + pupil + ct + (1 | name) 
\end{verbatim}

In this dataset all the variables are level-1 (patient), except by the
level-2 (study) type variable. The analysis model for this dataset is a
prediction model with \texttt{mort} as the outcome. In this tutorial
we'll estimate the adjusted prognostic effect of \texttt{ct} on
mortality outcomes. The estimand is the adjusted odds ratio for
\texttt{ct}, after including \texttt{type}, \texttt{age}
\texttt{motor\_score} and \texttt{pupil}. Therefore the main model for
the \(i\)-th patient from the \(j\)-th study can be described as:

\[ mort_{ij} = (\beta_0 + b_i) + \beta_1type_{.j}+ \beta_2age_{ij}+ \beta_3motorscore_{ij} + \beta_4pupil_{ij} + \beta_5ct_{ij}+\epsilon_{ij}\]
where \(\epsilon_{ij}\) corresponds to the error term.

Note that variables \texttt{hypots}, \texttt{hypox}, \texttt{tsah} and
\texttt{edh} are not part of the analysis model, and may thus serve as
auxiliary variables for imputation.

The \texttt{impact} data included in the \pkg{metamisc} package is a
complete data set. The original data has already been imputed once
(Steyerberg et al, 2008). For the purpose of this tutorial we have
induced missingness (mimicking the missing data in the original data set
before imputation). The resulting incomplete data can be accessed from
\href{https://zenodo.com}{zenodo link to be created}.

Load the incomplete data into the R workspace:

\begin{verbatim}
R> dat <- read.table("link/to/the/data.txt") 
\end{verbatim}

To explore the missingness, we should look at the missing data pattern.
The ten most frequent missingness patterns are shown with

\begin{verbatim}
R> plot_pattern(dat, rotate = TRUE, npat = 10L)  
\end{verbatim}

\begin{figure}[h]

{\centering \includegraphics{manuscript_files/figure-pdf/unnamed-chunk-27-1.pdf}

}

\end{figure}

This shows that we need to impute \texttt{ct} and \texttt{pupil}. To
develop the best imputation model, we need to investigate the relations
between the observed values of the incomplete variables and the observed
values of other variables, and the relation between the missingness
indicators of the incomplete variables and the observed values of the
other variables. To see whether the missingness depends on the observed
values of other variables, we can test this statistically or use visual
inspection (e.g.~a histogram faceted by the missingness indicator).

We should impute the variables \texttt{ct} and \texttt{pupil} and any
auxiliary variables we might want to use to impute these incomplete
analysis model variables. We can evaluate which variables may be useful
auxiliaries by plotting the pairwise complete correlations

\begin{verbatim}
R> plot_corr(dat, rotate = TRUE)
\end{verbatim}

\begin{figure}[h]

{\centering \includegraphics{manuscript_files/figure-pdf/unnamed-chunk-28-1.pdf}

}

\end{figure}

This shows us that \texttt{hypox} and \texttt{hypot} would not be useful
auxiliary variables for imputing \texttt{ct}. Depending on the minimum
required correlation, \texttt{tsah} could be useful, while \texttt{edh}
has the strongest correlation with \texttt{ct} out of all the variables
in the data and should definitely be included in the imputation model.
For the imputation of \texttt{pupil}, none of the potential auxiliary
variables has a very strong relation, but \texttt{hypots} could be used.
We conclude that we can exclude \texttt{hypox} from the data, since this
is neither an analysis model variable nor an auxiliary variable for
imputation

\begin{verbatim}
R> dat <- select(dat, !hypox)
\end{verbatim}

Mutate data to get the right data types for imputation (e.g.~integer for
clustering variable).

\begin{verbatim}
R> # dat <- mutate(
R> #   dat,
R> #   name = as.integer(name))
\end{verbatim}

This is necessary because otherwise PMM cannot be used for these factor
variables.

\begin{verbatim}
R> dat <- mutate(
+  dat,
+  across(everything(), as.numeric))
\end{verbatim}

Create an initial methods vector for the incomplete variables

\begin{verbatim}
R> meth <- make.method(dat)
R> meth
\end{verbatim}

\begin{verbatim}
       name        type         age motor_score       pupil          ct 
         ""          ""          ""          ""       "pmm"       "pmm" 
     hypots        tsah         edh        mort 
      "pmm"       "pmm"       "pmm"          "" 
\end{verbatim}

which should be adjusted to the appropriate \texttt{2l} methods.

\begin{verbatim}
R> # meth[c("pupil", "ct")] <- "2l.pmm"
R> # meth[c("hypots", "tsah", "edh")] <- "2l.pmm"
R> meth[meth == "pmm"] <- "2l.pmm"
\end{verbatim}

Create an initial predictor matrix

\begin{verbatim}
R> pred <- quickpred(dat)
\end{verbatim}

This predictor matrix is too large to display inline. A visualization of
the adapted predictor matrix is presented in Figure XYZ.

We should make sure \texttt{name} is used as clustering variable

\begin{verbatim}
R> pred[, "name"] <- -2
\end{verbatim}

and the analysis-model outcome should be used as a predictor in all
imputation models

\begin{verbatim}
R> pred[, "mort"] <- 2
R> # pred[pred == 1] <- 2
\end{verbatim}

the resulting predictor matrix is visualized with

\begin{verbatim}
R> plot_pred(pred, method = meth, rotate = TRUE)
\end{verbatim}

\begin{figure}[h]

{\centering \includegraphics{manuscript_files/figure-pdf/unnamed-chunk-37-1.pdf}

}

\end{figure}

Impute the incomplete data with

\begin{verbatim}
R> imp <- mice(
+  dat,
+  method = meth,
+  predictorMatrix = pred,
+  printFlag = FALSE
+)
\end{verbatim}

Evaluate the convergence of the algorithm

\begin{verbatim}
R> plot_trace(imp)
\end{verbatim}

\begin{figure}[h]

{\centering \includegraphics{manuscript_files/figure-pdf/unnamed-chunk-39-1.pdf}

}

\end{figure}

Evaluate the imputed values.

\begin{verbatim}
R> ggmice(imp, aes(.imp, pupil)) +
+  geom_jitter()
\end{verbatim}

\begin{figure}[h]

{\centering \includegraphics{manuscript_files/figure-pdf/unnamed-chunk-40-1.pdf}

}

\end{figure}

Convert the data back to factors.

\begin{verbatim}
long <- complete(imp, "long", include = TRUE)
long <- mutate(long, across(c("motor_score", "pupil", "ct"), as.factor))
imp <- as.mids(long)
\end{verbatim}

Analyze the imputed data:

\begin{verbatim}
R> fit <- imp %>%
+  with(glmer(
+    mort ~ type + age + motor_score + pupil + ct + (1 | name),
+    family = "binomial"
+    ))
\end{verbatim}

The estimated effects after imputation are presented in Table XYZ.

\begin{verbatim}
R> est <- testEstimates(as.mitml.result(fit), extra.pars = TRUE)
\end{verbatim}

Display results in table.

\begin{verbatim}
R> est$estimates |> 
+  round(3) |>
+  kable()
\end{verbatim}

\begin{longtable}[]{@{}
  >{\raggedright\arraybackslash}p{(\columnwidth - 14\tabcolsep) * \real{0.1605}}
  >{\raggedleft\arraybackslash}p{(\columnwidth - 14\tabcolsep) * \real{0.1111}}
  >{\raggedleft\arraybackslash}p{(\columnwidth - 14\tabcolsep) * \real{0.1235}}
  >{\raggedleft\arraybackslash}p{(\columnwidth - 14\tabcolsep) * \real{0.0988}}
  >{\raggedleft\arraybackslash}p{(\columnwidth - 14\tabcolsep) * \real{0.1358}}
  >{\raggedleft\arraybackslash}p{(\columnwidth - 14\tabcolsep) * \real{0.2222}}
  >{\raggedleft\arraybackslash}p{(\columnwidth - 14\tabcolsep) * \real{0.0741}}
  >{\raggedleft\arraybackslash}p{(\columnwidth - 14\tabcolsep) * \real{0.0741}}@{}}
\toprule()
\begin{minipage}[b]{\linewidth}\raggedright
\end{minipage} & \begin{minipage}[b]{\linewidth}\raggedleft
Estimate
\end{minipage} & \begin{minipage}[b]{\linewidth}\raggedleft
Std.Error
\end{minipage} & \begin{minipage}[b]{\linewidth}\raggedleft
t.value
\end{minipage} & \begin{minipage}[b]{\linewidth}\raggedleft
df
\end{minipage} & \begin{minipage}[b]{\linewidth}\raggedleft
P(\textgreater\textbar t\textbar)
\end{minipage} & \begin{minipage}[b]{\linewidth}\raggedleft
RIV
\end{minipage} & \begin{minipage}[b]{\linewidth}\raggedleft
FMI
\end{minipage} \\
\midrule()
\endhead
(Intercept) & -1.910 & 0.335 & -5.700 & 9617.768 & 0.000 & 0.021 &
0.021 \\
type & -0.347 & 0.179 & -1.941 & 256541.695 & 0.052 & 0.004 & 0.004 \\
age & 0.032 & 0.002 & 19.550 & 1965.628 & 0.000 & 0.047 & 0.046 \\
motor\_score2 & -0.576 & 0.070 & -8.195 & 14898.703 & 0.000 & 0.017 &
0.017 \\
motor\_score3 & -0.892 & 0.072 & -12.390 & 14760.294 & 0.000 & 0.017 &
0.017 \\
motor\_score4 & -1.272 & 0.074 & -17.203 & 10235.157 & 0.000 & 0.020 &
0.020 \\
pupil2 & 1.311 & 0.074 & 17.758 & 41.455 & 0.000 & 0.451 & 0.342 \\
pupil3 & 0.661 & 0.077 & 8.533 & 165.867 & 0.000 & 0.184 & 0.165 \\
ct2 & 0.783 & 0.075 & 10.461 & 264.558 & 0.000 & 0.140 & 0.130 \\
ct3 & 0.786 & 0.095 & 8.280 & 10.443 & 0.000 & 1.624 & 0.676 \\
\bottomrule()
\end{longtable}

\begin{verbatim}
R> est$extra.pars |> 
+  round(3) |>
+  kable()
\end{verbatim}

\begin{longtable}[]{@{}lr@{}}
\toprule()
& Estimate \\
\midrule()
\endhead
Intercept\textasciitilde\textasciitilde Intercept\textbar name &
0.083 \\
\bottomrule()
\end{longtable}

\hypertarget{obesity-data}{%
\subsection{Obesity data}\label{obesity-data}}

In this example, we demonstrate a multilevel imputation of random
intercept and random slope model with a continuous response. We utilize
the obesity dataset included in the \texttt{micemd}@ package, a
simulated dataset that emulates an electronic survey in which
individuals are asked to provide information about their weight and
consumption habits in different countries. We use the following
variables from the dataset:

\begin{itemize}
\tightlist
\item
  \textbf{Cluster:} Region of the patients' healthcare provider (Cluster
  variable),
\item
  \textbf{Gender:} Subjects' Gender (0=male, 1=female),
\item
  \textbf{Age:} Subjects' age,
\item
  \textbf{Height:} Subjects' height in metres,
\item
  \textbf{Weight:} Subjects' weight in kilograms,
\item
  \textbf{BMI:} Subjects' body mass index,
\item
  \textbf{FamOb:} Family obesity history (yes or no),
\item
  \textbf{Time:} Response time in minutes (exclusion-restriction
  variable).
\end{itemize}

In this dataset, Age and FamOb are MAR, while the weight variable is
affected by selection bias, attributed to an indirect MNAR mechanism.
This MNAR mechanism typically arises when an unobserved or omitted
variable influences both the value of the incomplete variable (in this
case, Weight) and its likelihood of being missing (denoted as R).

In the primary analysis model, BMI serves as the dependent variable,
with Age, Gender, and FamOb as predictors. Because of the clustered
nature of the data, which is quantified with the Intraclass Correlation
Coefficient (ICC) below, we include random intercepts, as well as a
random slope for the Age variable. The model is represented as:
\begin{equation}
\label{eqn:main}
BMI_{ij}= (\beta_{o}+ b_{oj} ) + (\beta_{1}+ b_{oj})* Age_{ij} + \beta_{2}*FamOb_{ij}+ \beta_{3}Gender_{ij} + \epsilon_{ij}
\end{equation}

We start by loading the data:

\begin{verbatim}
R> data(Obesity, package = "micemd")
\end{verbatim}

Now, let's begin by examining the missing patterns in the data by
cluster:

\begin{verbatim}
R> plot_pattern(Obesity)
\end{verbatim}

\begin{figure}[h]

{\centering \includegraphics{manuscript_files/figure-pdf/unnamed-chunk-47-1.pdf}

}

\end{figure}

\begin{verbatim}
R> # Obesity |> 
R> #   split(~Cluster) |>
R> #   lapply(plot_pattern)
\end{verbatim}

We observe that the missing pattern is non-monotonic and quite similar
across the clusters. However, regarding the weight variable, we notice
that it is systematically missing in cluster 3. In order to evaluate if
we require a imputation method that accounts for clustering we assess
the Intraclass Correlation

\begin{verbatim}
R> Nulmodel <- lme4::lmer(BMI ~ 1 + (1|Cluster), data = Obesity)
R> performance::icc(Nulmodel)
\end{verbatim}

\begin{verbatim}
# Intraclass Correlation Coefficient

    Adjusted ICC: 0.362
  Unadjusted ICC: 0.362
\end{verbatim}

As the ICC is greater than 0.1 and as we will use a mixed model for the
analysis, we decide to use two-level (2l) imputation methods. In this
imputation process, we include all predictor variables from equation
\ref{eqn:main} in the main model. However, since BMI is a composite of
weight and height, we use deterministic imputation for these, which is
described below.

We use the \textbf{find.defaultMethodfunction} provided in the
\textbf{micemd} package, which suggests an appropriate method for MAR
variables based on the type of variable, number of observations in the
cluster, and number of clusters.

It suggests using `2l.2stage.bin' for the FamOb variable and
`2l.2stage.norm' for the age variable. However, after inspecting the age
density plot, we consider modifying its method to `2l.2stage.pmm'. For
the BMI variable, we employ deterministic imputation.

\begin{verbatim}
meth_mar <-
  find.defaultMethod(
    Obesity,
    ind.clust = 1,
    I.small = 7,
    ni.small = 100,
    prop.small = 0.4
  )
meth_mar["BMI"] <- "~ I(Weight / (Height)^2)"
meth_mar["Age"] <- "2l.2stage.pmm"
meth_mar["Weight"] <- "2l.2stage.pmm"
\end{verbatim}

For these imputation models, it is necessary to specify the prediction
matrix, with the cluster variable labelled as -2 and the predictor
variable measured within clusters labelled as 2, encompassing all
variables. We need to supress the variable Time as this variable is not
specified in the main model. We also modify the relationship between
BMI, weight and height in the prediction matrix to avoid circular
predictions. Then we proceed to run the imputation model.

\begin{verbatim}
pred_mar <- quickpred(Obesity)
pred_mar["Weight",] <- 2
pred_mar["Weight","Weight"] <- 0
pred_mar[, "Cluster"] <- -2 
pred_mar[, "Time"] <- 0
pred_mar[pred_mar == 1] <- 2

pred_mar[c("Height", "Weight"), "BMI"] <- 0
#plot_pred(pred_mar)
imp_mar <-
  mice(
    data = Obesity,
    method = meth_mar,
    predictorMatrix = pred_mar,
    printFlag = FALSE
  )
\end{verbatim}

After confirming convergence, we proceed to save the results for future
use.

\begin{verbatim}
summary(complete(imp_mar, "long")$Weight)
\end{verbatim}

\begin{verbatim}
   Min. 1st Qu.  Median    Mean 3rd Qu.    Max. 
  28.35   69.38   81.96   82.36   94.46  134.61 
\end{verbatim}

\begin{verbatim}
plot_trace(imp_mar, "Weight")
\end{verbatim}

\begin{figure}[h]

{\centering \includegraphics{manuscript_files/figure-pdf/obesity-predmar_pmm1-1.pdf}

}

\end{figure}

\begin{verbatim}
ggmice(imp_mar, aes(.imp, Weight)) + 
  geom_jitter()
\end{verbatim}

\begin{figure}[h]

{\centering \includegraphics{manuscript_files/figure-pdf/obesity-predmar_pmm1-2.pdf}

}

\end{figure}

We consider the possibility that patients may not have been selected
randomly, which would then have led to a distribution for weight that
does not reflect the weight in the population. It's likely that an
omitted variable, like self-esteem, could influence this selection. For
instance, individuals with lower self-esteem might have higher weight
values, impacting their willingness to provide honest information due to
embarrassment.

To address this situation where data are Missing Not At Random (MNAR),
one approach is to apply the Heckman selection model. This method has
recently been extended to allow for variations in intercepts and
exposure effects (random intercept and slope)
\cite{galimard2018, munoz2023}. This method involves specifying two
equations: one for the outcome, describing the incomplete variable in
terms of partially observed predictors (in this case, all variables from
the main model), and the other for the selection model, explaining the
probability of being observed based (R) on certain variables.

To apply the \textbf{2l.2stage.heckman} method, the weight variable
should be specified as `2l.2stage.heckman' found in the micemd package.
Additionally, the prediction matrix needs modification, as this model
includes one submodel for the individuals' measured outcome, and one for
the selection of individuals. For the outcome equation we consider the
same imputation model that we used for the MAR case (main model).
\[Weight_{ij}= \beta^O_{o} + \beta^O_{1}Age_{ij} + \beta^O_{2}FamOb_{ij}+ \beta^O_{3}Gender_{ij} + \epsilon^O_{ij}\]
Regarding the selection equation, we include the same predictors as
those in the main model, as well as a time variable. Here the time
variable serves as a restriction exclusion variable specifically
explaining the probability of being observed but not affecting the
incomplete value (Weight). In this context, we assume that the time a
user spends completing the survey serves as a proxy for the barriers
they may encounter in survey completion, such as familiarity with the
survey content or internet speed. These factors may lead the user to
skip specific questions or even the entire survey. Also, we assume the
time does not have any influence on the subject's weight.
\[R_{ij}= \beta^S_{o} + \beta^S_{1}Age_{ij} + \beta^S_{2}FamOb_{ij}+ \beta^S_{3}Gender_{ij} +\beta^S_{4}Time_{ij}+ \epsilon^S_{ij}\]

These two equations are jointly estimated under the assumption that the
error terms are interconnected with a bivariate normal distribution. For
a more comprehensive understanding of the model and the exclusion
restriction, see \cite{munoz2023}.

To use information from both equations, we must adjust the prediction
matrix. The cluster variable remains specified as before (-2). In this
imputation method, all the variables present in both the selection and
outcome equations are included with a random effect.

However, it is essential to distinguish which of these variables appear
in each equation. In this framework, when a variable is shared between
both equations, it is denoted as (2). Predictors exclusive to the
outcome equation are indicated as (-4), while those exclusive to the
selection equation are labelled as (-3). Consequently, the only
alteration needed in the predictor matrix pertains to the variable
`Time'.

\begin{verbatim}
pred_mnar <- pred_mar
pred_mnar["Weight","Time"] <- -3
plot_pred(pred_mnar)
\end{verbatim}

\begin{figure}[h]

{\centering \includegraphics{manuscript_files/figure-pdf/obesity-predmnar-1.pdf}

}

\end{figure}

We also need to modify the method of the weight variable.

\begin{verbatim}
meth_mnar <- meth_mar
meth_mnar["Weight"] <- "2l.2stage.heckman"
\end{verbatim}

As before we use the `pmm', option but this time for the Heckman
imputation, this approach ensures that the imputed values remain within
the range of observable values. We then run the MNAR imputation model by
setting the pmm parameter to true.

\begin{verbatim}
imp_mnar <-
  mice(
    data = Obesity,
    method = meth_mnar,
    predictorMatrix = pred_mnar,
    pmm = T,
    printFlag = FALSE
  )
\end{verbatim}

We check the convergence of the results

\begin{verbatim}
summary(complete(imp_mnar, "long")$Weight)
\end{verbatim}

\begin{verbatim}
   Min. 1st Qu.  Median    Mean 3rd Qu.    Max. 
  28.35   67.39   82.57   81.71   96.04  134.61 
\end{verbatim}

\begin{verbatim}
plot_trace(imp_mar, "Weight")
\end{verbatim}

\begin{figure}[h]

{\centering \includegraphics{manuscript_files/figure-pdf/obesity-predmnarp1-1.pdf}

}

\end{figure}

We proceed to compare the effects on the model. We run the analysis
model on each of the completed datasets as well as the dataset where the
incomplete values are removed (Complete Case analysis, CC).

We proceed to fit the mixed model by using the complete case database
and the imputed datasets (MAR,MNAR), and we compare the estimates of
each dataset by using a plot (the plot function can be found in the
appendix).

\begin{verbatim}
# Save all datasets in a list
list_data <- list(Obesity |> filter(complete.cases(Obesity)), # complete case
                  imp_mar, # imputed under mar
                  imp_mnar) # imputed under mnar

# Run mixed models on each dataset
list_models <- lapply(list_data, 
                   FUN = function(x) 
                     with(x,nlme::lme(BMI ~ Age + FamOb + Gender, 
                                      random =  ~ 1 + Age | Cluster)))
# Plot coefficients
plot_models(list_models,
             mod_name = c("Complete case", "MAR", "MNAR"))
\end{verbatim}

\begin{figure}[h]

{\centering \includegraphics{manuscript_files/figure-pdf/models-1.pdf}

}

\end{figure}

We note that there is minimal disparity in the age effect, FamObs, or
Gender across the various imputation models under consideration. An
analysis of the intercept reveals that, under the MNAR assumption, a
higher average BMI is anticipated compared to the MAR assumption.
Nonetheless, with respect to precision of the estimates, we notice that
in general assuming MNAR leads to wider confidence intervals. In this
case it does not have any influence on the final result but there could
be cases where variation in the assumed missing mechanism could lead
also to differences on the significance of a statistical test and
therefore lead to different conclusions.

\hypertarget{conclusion}{%
\section{Conclusion}\label{conclusion}}

This paper is dedicated to exploring the imputation process for
incomplete datasets, with a primary focus on utilizing a hierarchical
model for analysis. Initially, users are encouraged to consider the main
analysis within the context of the incomplete dataset, along with
insights provided by domain experts, to gain a better understanding of
variable relationships. Employing clear data visualization is
instrumental in comprehending the missing data patterns, establishing a
missing mechanism, and aiding in the selection of suitable imputation
methods.

The ``Mice'' and ``Mice''-based R packages offer a range of imputation
methods tailored for hierarchical data, easily adaptable to the
dataset's structure. Before proceeding with the analysis of the imputed
dataset, it is essential to assess the convergence of the imputation
method. This evaluation can reveal issues such as circular problems, the
need for additional iterations, or challenges associated with the chosen
imputation method.

\hypertarget{sec-summary}{%
\section{Summary and discussion}\label{sec-summary}}

What is missing from this manuscript\ldots{}

\hypertarget{computational-details}{%
\section*{Computational details}\label{computational-details}}
\addcontentsline{toc}{section}{Computational details}

The results in this paper were obtained using
\proglang{R}\textasciitilde4.3.0. \proglang{R} itself and all packages
used are available from the Comprehensive \proglang{R} Archive Network
(CRAN) at {[}https://CRAN.R-project.org/{]}.

\hypertarget{disclaimer}{%
\section{Disclaimer}\label{disclaimer}}

The views expressed in this paper are the personal views of the authors
and may not be understood or quoted as being made on behalf of or
reflecting the position of the regulatory agency/agencies or
organizations with which the authors are employed/affiliated.

\hypertarget{acknowledgments}{%
\section*{Acknowledgments}\label{acknowledgments}}
\addcontentsline{toc}{section}{Acknowledgments}

This project has received funding from the European Union's Horizon 2020
research and innovation programme under ReCoDID grant agreement No
825746.

\hypertarget{references}{%
\section*{References}\label{references}}
\addcontentsline{toc}{section}{References}

\hypertarget{refs}{}
\begin{CSLReferences}{0}{0}
\end{CSLReferences}

\newpage{}

\hypertarget{sec-techdetails}{%
\section*{More technical details}\label{sec-techdetails}}
\addcontentsline{toc}{section}{More technical details}

\begin{tcolorbox}[enhanced jigsaw, opacityback=0, breakable, bottomrule=.15mm, colback=white, left=2mm, colframe=quarto-callout-color-frame, rightrule=.15mm, leftrule=.75mm, arc=.35mm, toprule=.15mm]

Appendices can be included after the bibliography (with a page break).
Each section within the appendix should have a proper section title
(rather than just \emph{Appendix}).

For more technical style details, please check out JSS's style FAQ at
{[}https://www.jstatsoft.org/pages/view/style\#frequently-asked-questions{]}
which includes the following topics:

\begin{itemize}
\tightlist
\item
  Title vs.~sentence case.
\item
  Graphics formatting.
\item
  Naming conventions.
\item
  Turning JSS manuscripts into \proglang{R} package vignettes.
\item
  Trouble shooting.
\item
  Many other potentially helpful details\ldots{}
\end{itemize}

\end{tcolorbox}

\hypertarget{sec-bibtex}{%
\section*{Using BibTeX}\label{sec-bibtex}}
\addcontentsline{toc}{section}{Using BibTeX}

\begin{tcolorbox}[enhanced jigsaw, opacityback=0, breakable, bottomrule=.15mm, colback=white, left=2mm, colframe=quarto-callout-color-frame, rightrule=.15mm, leftrule=.75mm, arc=.35mm, toprule=.15mm]

References need to be provided in a \textsc{Bib}{\TeX} file
(\texttt{.bib}). All references should be made with \texttt{@cite}
syntax. This commands yield different formats of author-year citations
and allow to include additional details (e.g.,pages, chapters, \dots) in
brackets. In case you are not familiar with these commands see the JSS
style FAQ for details.

Cleaning up \textsc{Bib}{\TeX} files is a somewhat tedious task --
especially when acquiring the entries automatically from mixed online
sources. However, it is important that informations are complete and
presented in a consistent style to avoid confusions. JSS requires the
following format.

\begin{itemize}
\tightlist
\item
  item JSS-specific markup (\texttt{\textbackslash{}proglang},
  \texttt{\textbackslash{}pkg}, \texttt{\textbackslash{}code}) should be
  used in the references.
\item
  item Titles should be in title case.
\item
  item Journal titles should not be abbreviated and in title case.
\item
  item DOIs should be included where available.
\item
  item Software should be properly cited as well. For \proglang{R}
  packages \texttt{citation("pkgname")} typically provides a good
  starting point.
\end{itemize}

\end{tcolorbox}


  \bibliography{bibliography.bib}


\end{document}
