\documentclass[
]{jss}

%% recommended packages
\usepackage{orcidlink,thumbpdf,lmodern}

\usepackage[utf8]{inputenc}

\author{
Hanne I. Oberman\\Methodology and Statistics\\
Utrecht University \And Johanna Munoz\\Julius Center for Health Sciences
and Primary Care,\\
University Medical Center Utrecht, Utrecht University,\\
Utrecht, The Netherlands \AND Thomas P. A. Debray\\Julius Center
for Health Sciences and Primary Care,\\
University Medical Center Utrecht, Utrecht University,\\
Utrecht, The Netherlands \And Gerko Vink\\Methodology and Statistics\\
Utrecht University \AND Valentijn M. T. de Jong\\Julius Center for
Health Sciences and Primary Care,\\
University Medical Center Utrecht, Utrecht University,\\
Utrecht, The Netherlands\\
Data Analytics and Methods Task Force,\\
European Medicines Agency,\\
Amsterdam, The Netherlands
}
\title{Imputation of Incomplete Multilevel Data with \pkg{mice}}

\Plainauthor{Hanne I. Oberman, Johanna Munoz, Thomas P. A. Debray, Gerko
Vink, Valentijn M. T. de Jong}
\Plaintitle{Imputation of Incomplete Multilevel Data with mice}
\Shorttitle{Multilevel \pkg{mice}}


\Abstract{
Incomplete multilevel data requires careful consideration of the missing
data problem and analysis strategy. In this tutorial, we focus on a
popular strategy for accommodating missingness in multilevel data:
replacing the missing data with one or more plausible values, i.e.,
imputation.Imputation separates the missing data problem from the main
analysis and the completed data can be analyzed as if it has been fully
observed.This tutorial illustrates the imputation of incomplete
multilevel data with the statistical programming language R. We aim to
show how imputation can yield less biased estimates from incomplete
clustered data. We provide practical guidelines and code snippets for
different missing data situations, including non-ignorable missingness
mechanisms. For brevity, we focus on multilevel imputation using chained
equations with the R mice package and its adjacent packages.
}

\Keywords{missing
data, multilevel, clustering, \pkg{mice}, \proglang{R}}
\Plainkeywords{missing data, multilevel, clustering, mice, R}

%% publication information
%% \Volume{50}
%% \Issue{9}
%% \Month{June}
%% \Year{2012}
%% \Submitdate{}
%% \Acceptdate{2012-06-04}

\Address{
    Hanne I. Oberman\\
    Methodology and Statistics\\
Utrecht University\\
    Padualaan 14\\
3584 CH Utrecht\\
  E-mail: \email{h.i.oberman@uu.nl}\\
  URL: \url{https://hanneoberman.github.io/}\\~\\
          }


% tightlist command for lists without linebreak
\providecommand{\tightlist}{%
  \setlength{\itemsep}{0pt}\setlength{\parskip}{0pt}}

% From pandoc table feature
\usepackage{longtable,booktabs,array}
\usepackage{calc} % for calculating minipage widths
% Correct order of tables after \paragraph or \subparagraph
\usepackage{etoolbox}
\makeatletter
\patchcmd\longtable{\par}{\if@noskipsec\mbox{}\fi\par}{}{}
\makeatother
% Allow footnotes in longtable head/foot
\IfFileExists{footnotehyper.sty}{\usepackage{footnotehyper}}{\usepackage{footnote}}
\makesavenoteenv{longtable}


\usepackage{graphicx}
\usepackage{mathtools}
\usepackage{ulem}

\usepackage{amsmath}

\begin{document}



\newpage

\hypertarget{introduction}{%
\section{Introduction}\label{introduction}}

Many datasets include individuals that are clustered together, for
example in geographic regions, or even different studies. In the
simplest case, individuals (e.g., students) are nested within a single
cluster (e.g., school classes). More complex clustered structures may
occur when there are multiple hierarchical levels (e.g., students in
different schools or patients within hospitals within regions across
countries), or when the clustering is non-nested (e.g., electronic
health record data from diverse settings and populations within large
databases). With clustered data we generally assume that individuals
from the same cluster tend to be more similar than individuals from
other clusters. In statistical terms, this implies that observations
from the same cluster are not independent and may in fact be correlated.
If this correlation is left unaddressed, estimates of \emph{p} values,
confidence intervals even model parameters are prone to bias
\citep{loca01}. Statistical methods for clustered data typically adopt
hierarchical models that explicitly describe the grouping of
observations. These models are also known as `multilevel models',
`hierarchical models', `mixed effect models', `random effect models',
and in the context of time-to-event data as `frailty models'. Table
\ref{tab:clus} provides an overview of some key concepts in multilevel
modeling.

\begin{CodeChunk}
\begin{table}

\caption{\label{tab:clus0}Concepts in multilevel methods}
\centering
\begin{tabular}[t]{>{\raggedright\arraybackslash}p{3cm}>{\raggedright\arraybackslash}p{12cm}}
\toprule
Concept & Details\\
\midrule
Sample units & Units of the population from which measurements are taken in a sample, e.g., students.\\
Cluster & Variable that specify the cluster or agruppation, e.g., Classroom\\
Hierarchical data & Data are grouped into clusters at different levels, observations belonging to the same cluster are expected to share certain characteristics.\\
Level-1 & Variable that varies within a cluster, eg. Test score\\
Level-2 & Variable that does not vary within a cluster but between, e.g. teacher experience.\\
\addlinespace
Hierarchical model & Model accounting for dependant observations relying on certain parameters ( within cluster) which in turn depend on other parameters (between cluster)\\
Fixed effect & Effects that are constant across all sample units, e.g. something that researchers control for and can repeat, such as  a teaching strategy (tutoring after class)\\
Random effect & Effects that are a source of random variation in the data, and whose levels are not fully sampled. e.g. test score tendency during academic year between students due to no controlled factors such as  genetic,family history\\
Mixed effect & Includes fixed and random effects, e.g. the fixed effect would be the treatment effect of a drug and the random effect would be the ID of the hospital where the patient is treated. Multilevel models typically accommodate for variability by including a separate group mean for each cluster e.g random intercept on hospitals. In addition to random intercepts, multilevel models can also include random coefficients and heterogeneous residual error variances across clusters (see e.g. @gelm06, @hox17 and @jong21).\\
ICC & The variability due to clustering is often measured by means of the intraclass coefficient (ICC). The ICC can be seen as the percentage
of variance that can be attributed to the cluster-level, where a high ICC would indicate that a lot of variability is due to the cluster structure.\\
\addlinespace
Stratified intercept & \\
\bottomrule
\end{tabular}
\end{table}

\end{CodeChunk}

\hypertarget{missingness-in-multilevel-data}{%
\subsection{Missingness in multilevel
data}\label{missingness-in-multilevel-data}}

As with any other dataset, clustered datasets may be impacted by
missingness in much the same way. Several strategies can be used to
handle missing data, including complete case analysis and imputation. We
focus on the latter approach and discuss statistical methods for
replacing the missing data with one or more plausible values. Imputation
separates the missing data problem from the analysis and the completed
data can be analyzed as if it were completely observed. It is generally
recommended to impute the missing values more than once to preserve
uncertainty due to missingness and to allow for valid inferences (c.f.
Rubin 1976).

With incomplete clustered datasets we can distinguish between two types
of missing data: sporadic missingness and systematic missingness
\citep{resc13}. Sporadic missingness arises when variables are missing
for some but not all of the units in a cluster \citep{buur18, jola18}.
For example, it is possible that test results are missing for several
students in one or more classes. When all observations are missing
within one or more clusters, data are said to be systematically missing.

\begin{CodeChunk}
\begin{figure}

{\centering \includegraphics{Imputation_of_Incomplete_Multilevel_Data_files/figure-latex/patterns-1} 

}

\caption[Sporadic missingness in multilevel data]{Sporadic missingness in multilevel data}\label{fig:patterns}
\end{figure}
\end{CodeChunk}

Imputation of missing data requires consideration of the mechanism
behind the missingness. Rubin proposed to distinguish between data that
are missing completely at random (MCAR), data that are missing at random
(MAR) and data that are missing not at random (MNAR; see Table
\ref{tab:miss}). For each of these three missingness generating
mechanisms, different imputation strategies are warranted
(\citet{yuce08} and \citet{hox15}). We here consider the general case
that data are MAR, and expand on certain MNAR situations.

\begin{CodeChunk}
\begin{table}

\caption{\label{tab:mechanism}Concepts in missing data methods}
\centering
\begin{tabular}[t]{>{\raggedright\arraybackslash}p{3cm}>{\raggedright\arraybackslash}p{12cm}}
\toprule
Concept & Details\\
\midrule
MCAR & Missing Completely At Random, where the probability to be missing is equal across all data entries.\\
MAR & Missing At Random, where the probability to be missing depends on observed information\\
MNAR & Missing Not At Random (MNAR), where the probability to be missing depends on unrecorded information, making the missingness non-ignorable [@rubi76; @meng94]\\
\bottomrule
\end{tabular}
\end{table}

\end{CodeChunk}

\hypertarget{aim-of-this-paper}{%
\subsection{Aim of this paper}\label{aim-of-this-paper}}

This papers serves as a tutorial for imputing incomplete multilevel data
with \pkg{mice} in \proglang{R}. \pkg{mice} has become the de-facto
standard for imputation by chained equations, which iteratively solves
the missingness on a variable-by-variable basis. \pkg{mice} is known to
yield valid inferences under many different missing data circumstances
\citep{buur18}.

We provide practical guidelines and code snippets for different missing
data situations, including non-ignorable mechanisms. For reasons of
brevity, we focus on multilevel imputation by chained equations with
\pkg{mice} exclusively; other imputation methods and packages \citep[see
e.g.][ and \citet{grun18}]{audi18} are outside the scope of this
tutorial. Assumed knowledge includes basic familiarity with the
\pkg{lme4} notation for multilevel models (see Table \ref{tab:mod}).

We illustrate imputation of incomplete multilevel data using three case
studies:

\begin{itemize}
\tightlist
\item
  \texttt{popmis} from the \pkg{mice} package \citep[simulated data on
  perceived popularity, \(n = 2,000\) pupils across \(N = 100\) schools
  with data that are MAR,][]{mice};
\item
  \texttt{impact} from the \pkg{metamisc} package \citep[empirical data
  on traumatic brain injuries, \(n = 11,022\) patients across \(N = 15\)
  studies with data that are MAR,][]{metamisc};
\item
  \texttt{obesity} from the \pkg{micemd} package {[}simulated data on
  obesity, \(n = 2,111\) patients across \(N = 5\) regions with data
  that are MNAR{]}.
\end{itemize}

For each of these datasets, we discuss the nature of the missingness,
choose one or more imputation models and evaluate the imputed data, but
we will also highlight one specific aspect of the imputation workflow.

This tutorial is dedicated to readers who are unfamiliar with multiple
imputation. More experienced readers can skip the introduction (case
study 1) and directly head to practical applications of multilevel
imputation under MAR conditions (case study 2) or under MNAR conditions
(case study 3).

\hypertarget{imputation-workflow}{%
\subsection{Imputation workflow}\label{imputation-workflow}}

Below we provide a imputation workflow that can be used in general to
impute cluster data.

\hypertarget{main-analysis}{%
\subsubsection{Main analysis}\label{main-analysis}}

When imputing clustered data, our first consideration should be the
research question and the type of analysis that researchers intend to
perform, assuming that non-incomplete values were present. This initial
assessment provides valuable insights into the data's structure (see the
level table), relationships among variables, and the causes of
missingness.

When multiple imputation is used to deal with missing data, as the
imputation and analysis process is performed separately, it is necessary
that imputation model being congenial with the main analysis model
(Meng, 1994), e.g.~if the main model accounts for the hierarchical
structure also imputation model should do it (Audigier, 2021).

\hypertarget{exploration-of-available-data}{%
\subsubsection{Exploration of available
data}\label{exploration-of-available-data}}

Then it is required to explore available information across clusters.
This can involve evaluating the Intra-class Correlation Coefficient
(ICC) to determine if a one-level imputation method is sufficient,
particularly if there are no substantial differences between clusters.
Additionally, it can be done by visualizing the data using available
package as \textbf{ggmice}, for instance examining density plots can
help us understand the variable type (normal, Poisson, categorical) and
its range of plausible values. Exploring relationships between
covariates through bivariate plots can be valuable in identifying
potential interactions or non-linearities.

In the context of missingness, we should assess patterns of systematic
and sporadic missing values and explore the relationships between the
missingness of an incomplete variable and observable data. This
evaluation allows us to consider the plausibility of Missing at Random
(MAR) mechanisms over Missing Completely at Random (MCAR) methods and
helps in selecting prediction variables to be included in the imputation
model (considering overflux).

\hypertarget{selecting-plausible-imputation-methods}{%
\subsubsection{Selecting plausible imputation
methods}\label{selecting-plausible-imputation-methods}}

Before to implement multiple imputation it is requiered to evaluate if
simpler imputation methods are suffice: - For example, certain
incomplete variables may not necessitate stochastic imputation methods
like MICE. Instead, they can be effectively addressed through deductive
imputation, where incomplete values can be inferred from logic and
deterministic relationships between variables. This approach is
particularly useful for variables that are functions of others, e.g.~a
person's BMI from their weight and height, determining values for
one-level variables from two-level ones, e.g.~in the context of
Individual Participant Data (IPD), incomplete information can be
extracted from metadata, such as deducing incomplete data about abortion
in a country where abortion is illegal, or engaging in cross-temporal or
protocol's deduction, like imputing missing test values for deceased
patients.

\begin{itemize}
\item
  Under monotonic pattern, the use of regression models
\item
  CC when the proportion of missing is minimal \textless5\%.
\item
  Analysis that are robust to missing data, for instance FIML models
  such as mixed model do not require imputation models when only the
  outcome variable is missing.An alternative approach to missing data is
  to use Full Information Maximum Likelihood (FIML). This method does
  not require the imputation of any missing values. Whereas MI consists
  of imputation, analyses and pooling steps, FIML analyses the data in a
  single step. When the assumptions are met the two approaches should
  produce equivalent results. {[}REF{]} As FIML requires specialised
  software, not all analyses can be performed with standard software.
  {[}REF{]}
\item
  Add table describing 2l- imputation models. like this
  \url{https://bookdown.org/mwheymans/bookmi/multiple-imputation-models-for-multilevel-data.html}
\end{itemize}

\hypertarget{imputation-process}{%
\subsubsection{Imputation process}\label{imputation-process}}

\hypertarget{clustering-specification}{%
\paragraph{Clustering specification}\label{clustering-specification}}

There are different strategies that can be adopted in the imputation
process that account for clustering: inclusion of cluster indicator
variable, performing a separate imputation process for each cluster, or
performing a simultaneous imputation process by using an imputation
method that accounts for clustering.(Stata:
\url{https://www.stata.com/support/faqs/statistics/clustering-and-mi-impute/})
TODO: replace ref.

The selection of each strategy depends mainly on the assumptions in the
main analysis and also on the restriction of the analyzed data.

Regarding the restrictions imposed by the data, for instance, the use of
cluster indicator variables is restricted in datasets where there are
not many clusters and many observations per cluster (Graham, 2009). The
last restriction is also required when imputations are performed on each
cluster separately. When this restriction cannot be achieved, one can
use an imputation model that simultaneously imputes all clusters using a
hierarchical model (Allison 2002).

Under this hierarchical imputation model, observations within clusters
are correlated and this correlation is modeled by a random effect so the
hierarchical model can be estimated even when there are few observations
per cluster. However, this strategy is best suited for balanced data
(Grund, 2017) and when random effects model is appropriated, i.e.~the
number of clusters is adequate. (Austin,2018).

Here it is important to evaluate the assumptions imposed by the main
model, for instance by using the cluster indicator strategy may lead to
bias estimates when the model is based on a hierarchical model
(Taaljard,2008). Even when an imputation strategy congenial with the
main model is preferred, it is important to consider whether it is
appropriate for the data as a less complex imputation strategies may
also lead to unbiased estimates in certain scenarios(Bailey 2020). For
instance, in causal effect analysis, separately imputation may lead to
smaller bias when the size of the smaller exposure cluster is large,
compared with an imputation model that includes exposure-confounder
interactions. (Zhang,2023).

\hypertarget{selection-of-variables-in-the-imputation-model}{%
\paragraph{Selection of variables in the imputation
model}\label{selection-of-variables-in-the-imputation-model}}

\begin{itemize}
\tightlist
\item
  Inclusion of outcome variable {[}Moons{]} When the outcome is
  time-to-event, the Nelson-Aalen estimate of the time to event should
  be included as a covariate in the imputation model {[}REF{]}
\item
  Imputation for interaction term, JAV, Machine lerarning methods
  (Random forest) and imputation separated by treatment group
\item
  Prediction matrix specfication (table)
\end{itemize}

\hypertarget{convergency-problems-during-imputation}{%
\paragraph{Convergency problems during
imputation}\label{convergency-problems-during-imputation}}

\begin{itemize}
\tightlist
\item
  Check log
\item
  Problems of convergence on some 2l methods can be solved by use of
  overflux or quickpred.
\item
  Consider the use of scaled variables Vanburen.
\item
  Consider to swap imputation mehtod in problematic variables: The use
  of non parametrical imputation methods such as pmm. or downgrade to 1l
  methods.
\item
  Also by the specification of post-processing in some variables can
  alivianate this.
\item
  Separated imputation process for each endpoint analysis
\end{itemize}

\hypertarget{convergency}{%
\subsubsection{Convergency}\label{convergency}}

Plots and also the prediction plots.

\hypertarget{sensibility-analysis}{%
\subsubsection{Sensibility analysis}\label{sensibility-analysis}}

About the possibility of assumptions about Missing mechanism.

\hypertarget{setup}{%
\subsection{Setup}\label{setup}}

Install non-CRAN packages if necessary:

\begin{CodeChunk}
\begin{CodeInput}
R> devtools::install_github("amices/ggmice")
\end{CodeInput}
\end{CodeChunk}

Set up the R environment and load the necessary packages:

\begin{CodeChunk}
\begin{CodeInput}
R> set.seed(2022)        # for reproducibility
R> library(mice)         # for imputation
R> library(miceadds)     # for additional imputation routines
R> library(ggmice)       # for incomplete/imputed data visualization
R> library(ggplot2)      # for visualization
R> library(dplyr)        # for data wrangling
R> library(lme4)         # for multilevel modeling
R> library(mitml)        # for multilevel parameter pooling
R> library(micemd)  # for imputation cf. heckman models
R> library(metamisc)     # for case study data
\end{CodeInput}
\end{CodeChunk}

\hypertarget{case-study-i-popularity-data}{%
\section{Case study I: popularity
data}\label{case-study-i-popularity-data}}

In this section we will go over the different steps involved with
imputing incomplete multilevel data with the R package mice. We consider
the simulated \texttt{popmis} dataset, which included pupils
(\(n = 2000\)) clustered within schools (\(N = 100\)). The following
variables are of primary interest:

\begin{itemize}
\tightlist
\item
  \texttt{school}, school identification number (clustering variable);
\item
  \texttt{popular}, pupil popularity (self-rating between 0 and 10;
  unit-level);
\item
  \texttt{sex}, pupil sex (0=boy, 1=girl; unit-level);
\item
  \texttt{texp}, teacher experience (in years; cluster-level).
\end{itemize}

The research objective of the \texttt{popmis} dataset is to predict the
pupils' popularity based on their gender and the experience of the
teacher. The analysis model corresponding to this dataset is multilevel
regression with random intercepts, random slopes and a cross-level
interaction. The outcome variable is \texttt{popular}, which is
predicted from the unit-level variable \texttt{sex} and the
cluster-level variable \texttt{texp}:

\begin{CodeChunk}
\begin{CodeInput}
R> mod <- popular ~ 1 + sex + (1 | school)
\end{CodeInput}
\end{CodeChunk}

Load the data into the environment and select the relevant variables:

\begin{CodeChunk}
\begin{CodeInput}
R> popmis <- popmis[, c("school", "popular", "sex")] 
\end{CodeInput}
\end{CodeChunk}

First we plot the pattern of missing data within categories of the
relevant variables. Plot the missing data pattern:

\begin{CodeChunk}
\begin{CodeInput}
R> plot_pattern(popmis)
\end{CodeInput}
\begin{figure}

{\centering \includegraphics{Imputation_of_Incomplete_Multilevel_Data_files/figure-latex/pop_pat-1} 

}

\caption[Missing data pattern in the popularity data]{Missing data pattern in the popularity data}\label{fig:pop_pat}
\end{figure}
\end{CodeChunk}

The missingness is univariate and sporadic, which is illustrated in the
missing data pattern in Figure \ref{fig:pop_pat}.

To develop the best imputation model for the incomplete variable
\texttt{popular}, we need to know whether the observed values of
\texttt{popular} are related to observed values of other variables. Plot
the pair-wise complete correlations in the incomplete data:

\begin{CodeChunk}
\begin{CodeInput}
R> plot_corr(popmis)
\end{CodeInput}


\begin{center}\includegraphics{Imputation_of_Incomplete_Multilevel_Data_files/figure-latex/pop-corr-1} \end{center}

\end{CodeChunk}

This shows us that \texttt{sex} may be a useful imputation model
predictor. Moreover, the missingness in \texttt{popular} may depend on
the observed values of other variables.

\begin{CodeChunk}
\begin{CodeInput}
R> # ggmice(popmis, aes(sex)) +
R> #   geom_histogram(fill = "white") +
R> #   facet_grid(. ~ is.na(popular), scales = "free", labeller = label_both)
R> 
R> ggplot(popmis, aes(y = popular, group = sex)) +
+   geom_boxplot() + 
+   theme_classic()
\end{CodeInput}


\begin{center}\includegraphics{Imputation_of_Incomplete_Multilevel_Data_files/figure-latex/pop-hist-1} \end{center}

\end{CodeChunk}

\hypertarget{imputation-ignoring-the-cluster-variable-not-recommended}{%
\subsubsection{Imputation ignoring the cluster variable (not
recommended)}\label{imputation-ignoring-the-cluster-variable-not-recommended}}

The first imputation model that we'll use is likely to be invalid. We do
\emph{not} use the cluster identifier \texttt{school} as imputation
model predictor. With this model, we ignore the multilevel structure of
the data, despite the high ICC. This assumes exchangeability between
units. We include it purely to illustrate the effects of ignoring the
clustering in our imputation effort.

Create a methods vector and predictor matrix for \texttt{popular}, and
make sure \texttt{school} is not included as predictor:

\begin{CodeChunk}
\begin{CodeInput}
R> meth <- make.method(popmis) # methods vector
R> pred <- quickpred(popmis)   # predictor matrix
R> plot_pred(pred)
\end{CodeInput}


\begin{center}\includegraphics{Imputation_of_Incomplete_Multilevel_Data_files/figure-latex/pop-ignored-pred-1} \end{center}

\end{CodeChunk}

Impute the data, ignoring the cluster structure:

\begin{CodeChunk}
\begin{CodeInput}
R> imp <- mice(popmis, pred = pred, print = FALSE)
\end{CodeInput}
\end{CodeChunk}

Analyze the imputations:

\begin{CodeChunk}
\begin{CodeInput}
R> fit <- with(imp, 
+             lmer(popular ~ 1 + sex  + (1 | school))) 
\end{CodeInput}
\end{CodeChunk}

Print the estimates:

\begin{CodeChunk}
\begin{CodeInput}
R> testEstimates(as.mitml.result(fit), extra.pars = TRUE)
\end{CodeInput}
\begin{CodeOutput}

Call:

testEstimates(model = as.mitml.result(fit), extra.pars = TRUE)

Final parameter estimates and inferences obtained from 5 imputed data sets.

             Estimate Std.Error   t.value        df   P(>|t|)       RIV       FMI 
(Intercept)     5.012     0.295    16.994     4.362     0.000    22.587     0.969 
sex             0.695     0.251     2.768     4.287     0.047    28.390     0.975 

                            Estimate 
Intercept~~Intercept|school    0.266 
Residual~~Residual             1.035 
ICC|school                     0.208 

Unadjusted hypothesis test as appropriate in larger samples.
\end{CodeOutput}
\end{CodeChunk}

\hypertarget{imputation-with-the-cluster-variable-as-predictor-not-recommended}{%
\subsubsection{Imputation with the cluster variable as predictor (not
recommended)}\label{imputation-with-the-cluster-variable-as-predictor-not-recommended}}

We'll now use \texttt{school} as a predictor to impute all other
variables. This is still not recommended practice, since it only works
under certain circumstances and results may be biased
\citep{drec15, ende16}. But at least, it includes some multilevel
aspect. This method is also called `fixed cluster imputation', and uses
N-1 indicator variables representing allocation of N clusters as a fixed
factor in the model \citep{reit06, ende16}. Colloquially, this is
`multilevel imputation for dummies'.

\begin{CodeChunk}
\begin{CodeInput}
R> # adjust the predictor matrix
R> pred["popular", "school"] <- 1 
R> plot_pred(pred)
\end{CodeInput}


\begin{center}\includegraphics{Imputation_of_Incomplete_Multilevel_Data_files/figure-latex/pop_predictor-1} \end{center}

\begin{CodeInput}
R> # impute the data, cluster as predictor
R> imp <- mice(popmis, pred = pred, print = FALSE)
\end{CodeInput}
\end{CodeChunk}

Analyze the imputations:

\begin{CodeChunk}
\begin{CodeInput}
R> fit <- with(imp, 
+             lmer(popular ~ 1 + sex + (1 | school))) 
\end{CodeInput}
\end{CodeChunk}

Print the estimates:

\begin{CodeChunk}
\begin{CodeInput}
R> testEstimates(as.mitml.result(fit), extra.pars = TRUE)
\end{CodeInput}
\begin{CodeOutput}

Call:

testEstimates(model = as.mitml.result(fit), extra.pars = TRUE)

Final parameter estimates and inferences obtained from 5 imputed data sets.

             Estimate Std.Error   t.value        df   P(>|t|)       RIV       FMI 
(Intercept)     4.915     0.217    22.642     4.926     0.000     9.110     0.926 
sex             0.975     0.283     3.444     4.250     0.024    32.504     0.978 

                            Estimate 
Intercept~~Intercept|school    0.351 
Residual~~Residual             1.153 
ICC|school                     0.233 

Unadjusted hypothesis test as appropriate in larger samples.
\end{CodeOutput}
\end{CodeChunk}

\hypertarget{imputation-with-multilevel-model}{%
\subsubsection{Imputation with multilevel
model}\label{imputation-with-multilevel-model}}

\begin{CodeChunk}
\begin{CodeInput}
R> # adjust the predictor matrix
R> pred["popular", "school"] <- -2 
R> plot_pred(pred)
\end{CodeInput}


\begin{center}\includegraphics{Imputation_of_Incomplete_Multilevel_Data_files/figure-latex/pop_multilevel-1} \end{center}

\begin{CodeInput}
R> # impute the data, cluster as predictor
R> imp <- mice(popmis, pred = pred, print = FALSE)
\end{CodeInput}
\end{CodeChunk}

Analyze the imputations:

\begin{CodeChunk}
\begin{CodeInput}
R> fit <- with(imp, 
+             lmer(popular ~ 1 + sex + (1 | school))) 
\end{CodeInput}
\end{CodeChunk}

Print the estimates:

\begin{CodeChunk}
\begin{CodeInput}
R> testEstimates(as.mitml.result(fit), extra.pars = TRUE)
\end{CodeInput}
\begin{CodeOutput}

Call:

testEstimates(model = as.mitml.result(fit), extra.pars = TRUE)

Final parameter estimates and inferences obtained from 5 imputed data sets.

             Estimate Std.Error   t.value        df   P(>|t|)       RIV       FMI 
(Intercept)     5.011     0.410    12.222     4.226     0.000    35.955     0.980 
sex             0.928     0.381     2.434     4.168     0.069    48.221     0.985 

                            Estimate 
Intercept~~Intercept|school    0.313 
Residual~~Residual             1.428 
ICC|school                     0.188 

Unadjusted hypothesis test as appropriate in larger samples.
\end{CodeOutput}
\end{CodeChunk}

\hypertarget{case-study-ii-impact-data-syst-missingness-pred-matrix}{%
\section{Case study II: IMPACT data (syst missingness, pred
matrix)}\label{case-study-ii-impact-data-syst-missingness-pred-matrix}}

We illustrate how to impute incomplete multilevel data by means of a
case study: \texttt{impact} from the \pkg{metamisc} package
\citep[empirical data on traumatic brain injuries, \(n = 11,022\) units
across \(N = 15\) clusters,][]{metamisc}. The \texttt{impact} data set
contains traumatic brain injury data on \(n = 11022\) patients clustered
in \(N = 15\) studies with the following 11 variables:

\begin{itemize}
\tightlist
\item
  \texttt{name} Name of the study,
\item
  \texttt{type} Type of study (RCT: randomized controlled trial, OBS:
  observational cohort),
\item
  \texttt{age} Age of the patient,
\item
  \texttt{motor\_score} Glasgow Coma Scale motor score,
\item
  \texttt{pupil} Pupillary reactivity,
\item
  \texttt{ct} Marshall Computerized Tomography classification,
\item
  \texttt{hypox} Hypoxia (0=no, 1=yes),
\item
  \texttt{hypots} Hypotension (0=no, 1=yes),
\item
  \texttt{tsah} Traumatic subarachnoid hemorrhage (0=no, 1=yes),
\item
  \texttt{edh} Epidural hematoma (0=no, 1=yes),
\item
  \texttt{mort} 6-month mortality (0=alive, 1=dead).
\end{itemize}

The analysis model for this dataset is a prediction model with
\texttt{mort} as the outcome. In this tutorial we'll estimate the
adjusted prognostic effect of \texttt{ct} on mortality outcomes. The
estimand is the adjusted odds ratio for \texttt{ct}, after including
\texttt{type}, \texttt{age} \texttt{motor\_score} and \texttt{pupil}
into the analysis model:

\begin{CodeChunk}
\begin{CodeInput}
R> mod <- mort ~ 1 + type + age + motor_score + pupil + ct + (1 | name) 
\end{CodeInput}
\end{CodeChunk}

Note that variables \texttt{hypots}, \texttt{hypox}, \texttt{tsah} and
\texttt{edh} are not part of the analysis model, and may thus serve as
auxiliary variables for imputation.

The \texttt{impact} data included in the \pkg{metamisc} package is a
complete data set. The original data has already been imputed once
(Steyerberg et al, 2008). For the purpose of this tutorial we have
induced missingness (mimicking the missing data in the original data set
before imputation). The resulting incomplete data can be accessed from
\href{https://zenodo.com}{zenodo link to be created}.

Load the complete and incomplete data into the R workspace:

\begin{CodeChunk}
\begin{CodeInput}
R> data("impact", package = "metamisc")      # complete data
R> dat <- read.table("link/to/the/data.txt") # incomplete data
\end{CodeInput}
\end{CodeChunk}

The estimated effects in the complete data are visualized in Figure
\ref{}.

\begin{CodeChunk}


\begin{center}\includegraphics{Imputation_of_Incomplete_Multilevel_Data_files/figure-latex/forest-1} \end{center}

\end{CodeChunk}

\begin{CodeChunk}
\begin{CodeInput}
R> # fit <- glmer(mod, family = "binomial", data = impact) # fit the model
R> # tidy(fit, conf.int = TRUE, exponentiate = TRUE)       # print estimates
\end{CodeInput}
\end{CodeChunk}

\hypertarget{missingness}{%
\subsection{Missingness}\label{missingness}}

To explore the missingness, it is wise to look at the missing data
pattern. The ten most frequent missingness patterns are shown:

\begin{CodeChunk}
\begin{CodeInput}
R> plot_pattern(dat, rotate = TRUE)  # plot missingness pattern
\end{CodeInput}


\begin{center}\includegraphics{Imputation_of_Incomplete_Multilevel_Data_files/figure-latex/pattern-1} \end{center}

\end{CodeChunk}

This shows that we need to impute \texttt{ct} and \texttt{pupil}.

To develop the best imputation model, we need to investigate the
relations between the observed values of the incomplete variables and
the observed values of other variables, and the relation between the
missingness indicators of the incomplete variables and the observed
values of the other variables. To see whether the missingness depends on
the observed values of other variables, we can test this statistically
or use visual inspection (e.g.~a histogram faceted by the missingness
indicator).

We should impute the variables \texttt{ct} and \texttt{pupil} and any
auxiliary variables we might want to use to impute these incomplete
analysis model variables. We can evaluate which variables may be useful
auxiliaries by plotting the pairwise complete correlations:

\begin{CodeChunk}
\begin{CodeInput}
R> plot_corr(dat, rotate = TRUE) # plot correlations 
\end{CodeInput}


\begin{center}\includegraphics{Imputation_of_Incomplete_Multilevel_Data_files/figure-latex/impact_corr-1} \end{center}

\end{CodeChunk}

This shows us that \texttt{hypox} and \texttt{hypot} would not be useful
auxiliary variables for imputing \texttt{ct}. Depending on the minimum
required correlation, \texttt{tsah} could be useful, while \texttt{edh}
has the strongest correlation with \texttt{ct} out of all the variables
in the data and should definitely be included in the imputation model.
For the imputation of \texttt{pupil}, none of the potential auxiliary
variables has a very strong relation, but \texttt{hypots} could be used.
We conclude that we can exclude \texttt{hypox} from the data, since this
is neither an analysis model variable nor an auxiliary variable for
imputation:

\begin{CodeChunk}
\begin{CodeInput}
R> dat <- select(dat, !hypox)  # remove variable
\end{CodeInput}
\end{CodeChunk}

\hypertarget{complete-case-analysis}{%
\subsection{Complete case analysis}\label{complete-case-analysis}}

As previously stated, complete case analysis lowers statistical power
and may bias results. The complete case analysis estimates are:

\begin{CodeChunk}
\begin{CodeInput}
R> fit <- glmer(mod, family = "binomial", data = na.omit(dat)) # fit the model
R> tidy(fit, conf.int = TRUE, exponentiate = TRUE)             # print estimates
\end{CodeInput}
\begin{CodeOutput}
# A tibble: 11 x 9
   effect  group term  estimate std.error statistic   p.value conf.low conf.high
   <chr>   <chr> <chr>    <dbl>     <dbl>     <dbl>     <dbl>    <dbl>     <dbl>
 1 fixed   <NA>  (Int~   0.0863   0.0182     -11.6   3.00e-31   0.0571     0.130
 2 fixed   <NA>  type~   0.757    0.137       -1.54  1.22e- 1   0.531      1.08 
 3 fixed   <NA>  age     1.03     0.00265     12.9   7.40e-38   1.03       1.04 
 4 fixed   <NA>  moto~   0.651    0.0732      -3.82  1.34e- 4   0.522      0.811
 5 fixed   <NA>  moto~   0.489    0.0555      -6.30  2.97e-10   0.391      0.611
 6 fixed   <NA>  moto~   0.274    0.0321     -11.0   2.28e-28   0.218      0.345
 7 fixed   <NA>  pupi~   3.20     0.317       11.7   8.18e-32   2.63       3.88 
 8 fixed   <NA>  pupi~   1.75     0.195        5.06  4.27e- 7   1.41       2.18 
 9 fixed   <NA>  ctIII   2.41     0.268        7.89  3.05e-15   1.94       2.99 
10 fixed   <NA>  ctIV~   2.30     0.214        8.95  3.56e-19   1.92       2.76 
11 ran_pa~ name  sd__~   0.230   NA           NA    NA         NA         NA    
\end{CodeOutput}
\end{CodeChunk}

As we can see, a higher \texttt{ct} (Marshall Computerized Tomography
classification) is associated with a lower odds of 6-month mortality,
given by the odds ratio exp(0.42), CI \ldots{} to \ldots, when
controlling for\ldots{}

\hypertarget{imputation-model}{%
\subsection{Imputation model}\label{imputation-model}}

Mutate data to get the right data types for imputation (e.g.~integer for
clustering variable).

\begin{CodeChunk}
\begin{CodeInput}
R> dat <- dat %>% mutate(across(everything(), as.integer))
\end{CodeInput}
\end{CodeChunk}

Create a methods vector and predictor matrix, and make sure
\texttt{name} is not included as predictor, but as clustering variable:

\begin{CodeChunk}
\begin{CodeInput}
R> meth <- make.method(dat) # methods vector
R> pred <- quickpred(dat)   # predictor matrix
R> plot_pred(pred, rotate = TRUE)
\end{CodeInput}


\begin{center}\includegraphics{Imputation_of_Incomplete_Multilevel_Data_files/figure-latex/impact-1} \end{center}

\begin{CodeInput}
R> pred[pred == 1] <- 2
R> pred["mort", ] <- 2
R> pred[, "mort"] <- 2
R> pred[c("name", "type", "age", "motor_score", "mort"), ] <- 0
R> pred[, "name"] <- -2
R> diag(pred) <- 0
R> plot_pred(pred, rotate = TRUE)
\end{CodeInput}


\begin{center}\includegraphics{Imputation_of_Incomplete_Multilevel_Data_files/figure-latex/impact-2} \end{center}

\begin{CodeInput}
R> meth <- make.method(dat)
R> meth
\end{CodeInput}
\begin{CodeOutput}
       name        type         age motor_score       pupil          ct 
         ""          ""          ""          ""       "pmm"       "pmm" 
     hypots        tsah         edh        mort 
      "pmm"       "pmm"       "pmm"          "" 
\end{CodeOutput}
\end{CodeChunk}

Impute the incomplete data

\begin{CodeChunk}
\begin{CodeInput}
R> imp <- mice(dat, method = meth, predictorMatrix = pred, printFlag = FALSE)
\end{CodeInput}
\end{CodeChunk}

\begin{CodeChunk}
\begin{CodeInput}
R> fit <- imp %>% 
+   with(glmer(mort ~ type + age + as.factor(motor_score) + pupil + ct + (1 | name), family = "binomial")) 
R> tidy(pool(fit))
\end{CodeInput}
\begin{CodeOutput}
                     term    estimate   std.error  statistic      p.value
1             (Intercept) -2.35203726 0.340181747  -6.914061 4.994037e-12
2                    type -0.41265892 0.180274846  -2.289054 2.209524e-02
3                     age  0.03049023 0.001570162  19.418521 1.238416e-81
4 as.factor(motor_score)2 -0.66764920 0.068737865  -9.712975 3.480413e-22
5 as.factor(motor_score)3 -1.05520001 0.070218940 -15.027285 2.540225e-50
6 as.factor(motor_score)4 -1.51238349 0.072304262 -20.916934 1.850073e-90
7                   pupil  0.48421447 0.038982800  12.421234 6.772069e-17
8                      ct  0.43474621 0.029968474  14.506785 2.342253e-36
             b          df dfcom         fmi      lambda m         riv
1 5.281599e-04 10119.70041 11013 0.005673265 0.005476771 5 0.005506932
2 4.881699e-05 10893.89378 11013 0.001985736 0.001802528 5 0.001805783
3 3.335358e-08  6320.89582 11013 0.016545467 0.016234340 5 0.016502243
4 4.188703e-05  8327.24771 11013 0.010875748 0.010638213 5 0.010752602
5 5.135721e-05  7632.20045 11013 0.012757637 0.012498967 5 0.012657168
6 1.403058e-04  2831.75462 11013 0.032888241 0.032205434 5 0.033277139
7 3.571497e-04    49.97295 11013 0.309130930 0.282023647 5 0.392803531
8 8.586645e-05   294.69765 11013 0.120677044 0.114729598 5 0.129598366
          ubar
1 1.150898e-01
2 3.244044e-02
3 2.425385e-06
4 4.674630e-03
5 4.869071e-03
6 5.059539e-03
7 1.091079e-03
8 7.950697e-04
\end{CodeOutput}
\begin{CodeInput}
R> as.mitml.result(fit)
\end{CodeInput}
\begin{CodeOutput}
[[1]]
Generalized linear mixed model fit by maximum likelihood (Laplace
  Approximation) [glmerMod]
 Family: binomial  ( logit )
Formula: mort ~ type + age + as.factor(motor_score) + pupil + ct + (1 |  
    name)
      AIC       BIC    logLik  deviance  df.resid 
10495.423 10561.192 -5238.712 10477.423     11013 
Random effects:
 Groups Name        Std.Dev.
 name   (Intercept) 0.2843  
Number of obs: 11022, groups:  name, 15
Fixed Effects:
            (Intercept)                     type                      age  
               -2.37195                 -0.41014                  0.03052  
as.factor(motor_score)2  as.factor(motor_score)3  as.factor(motor_score)4  
               -0.65802                 -1.04611                 -1.51245  
                  pupil                       ct  
                0.50405                  0.42496  
optimizer (Nelder_Mead) convergence code: 0 (OK) ; 0 optimizer warnings; 1 lme4 warnings 

[[2]]
Generalized linear mixed model fit by maximum likelihood (Laplace
  Approximation) [glmerMod]
 Family: binomial  ( logit )
Formula: mort ~ type + age + as.factor(motor_score) + pupil + ct + (1 |  
    name)
     AIC      BIC   logLik deviance df.resid 
10500.88 10566.65 -5241.44 10482.88    11013 
Random effects:
 Groups Name        Std.Dev.
 name   (Intercept) 0.2917  
Number of obs: 11022, groups:  name, 15
Fixed Effects:
            (Intercept)                     type                      age  
               -2.37718                 -0.41511                  0.03067  
as.factor(motor_score)2  as.factor(motor_score)3  as.factor(motor_score)4  
               -0.66935                 -1.05211                 -1.49429  
                  pupil                       ct  
                0.49013                  0.43835  
optimizer (Nelder_Mead) convergence code: 0 (OK) ; 0 optimizer warnings; 1 lme4 warnings 

[[3]]
Generalized linear mixed model fit by maximum likelihood (Laplace
  Approximation) [glmerMod]
 Family: binomial  ( logit )
Formula: mort ~ type + age + as.factor(motor_score) + pupil + ct + (1 |  
    name)
      AIC       BIC    logLik  deviance  df.resid 
10505.026 10570.795 -5243.513 10487.026     11013 
Random effects:
 Groups Name        Std.Dev.
 name   (Intercept) 0.2908  
Number of obs: 11022, groups:  name, 15
Fixed Effects:
            (Intercept)                     type                      age  
               -2.32339                 -0.42359                  0.03023  
as.factor(motor_score)2  as.factor(motor_score)3  as.factor(motor_score)4  
               -0.67142                 -1.05776                 -1.51038  
                  pupil                       ct  
                0.49756                  0.42474  
optimizer (Nelder_Mead) convergence code: 0 (OK) ; 0 optimizer warnings; 1 lme4 warnings 

[[4]]
Generalized linear mixed model fit by maximum likelihood (Laplace
  Approximation) [glmerMod]
 Family: binomial  ( logit )
Formula: mort ~ type + age + as.factor(motor_score) + pupil + ct + (1 |  
    name)
      AIC       BIC    logLik  deviance  df.resid 
10519.511 10585.280 -5250.755 10501.511     11013 
Random effects:
 Groups Name        Std.Dev.
 name   (Intercept) 0.2961  
Number of obs: 11022, groups:  name, 15
Fixed Effects:
            (Intercept)                     type                      age  
               -2.33581                 -0.40871                  0.03039  
as.factor(motor_score)2  as.factor(motor_score)3  as.factor(motor_score)4  
               -0.66477                 -1.05451                 -1.51858  
                  pupil                       ct  
                0.45928                  0.44419  
optimizer (Nelder_Mead) convergence code: 0 (OK) ; 0 optimizer warnings; 2 lme4 warnings 

[[5]]
Generalized linear mixed model fit by maximum likelihood (Laplace
  Approximation) [glmerMod]
 Family: binomial  ( logit )
Formula: mort ~ type + age + as.factor(motor_score) + pupil + ct + (1 |  
    name)
      AIC       BIC    logLik  deviance  df.resid 
10522.038 10587.807 -5252.019 10504.038     11013 
Random effects:
 Groups Name        Std.Dev.
 name   (Intercept) 0.2955  
Number of obs: 11022, groups:  name, 15
Fixed Effects:
            (Intercept)                     type                      age  
               -2.35187                 -0.40574                  0.03064  
as.factor(motor_score)2  as.factor(motor_score)3  as.factor(motor_score)4  
               -0.67468                 -1.06551                 -1.52622  
                  pupil                       ct  
                0.47006                  0.44148  
optimizer (Nelder_Mead) convergence code: 0 (OK) ; 0 optimizer warnings; 1 lme4 warnings 

attr(,"class")
[1] "mitml.result" "list"        
\end{CodeOutput}
\begin{CodeInput}
R> # testEstimates(as.mitml.result(fit))
\end{CodeInput}
\end{CodeChunk}

\hypertarget{case-study-iii-obesity-data}{%
\section{Case study III: obesity
data}\label{case-study-iii-obesity-data}}

In this example, we demonstrate a multilevel imputation of random
intercept and random slope model with a continuous response. We utilize
the obesity dataset included in the \texttt{micemd}@ package, a
simulated dataset that emulates an electronic survey in which
individuals are asked to provide information about their weight and
consumption habits in different countries. We simulate data for 5
clusters so that the true values are known. We use the following
variables from the dataset:

\begin{itemize}
\tightlist
\item
  \textbf{Cluster:} Region of the patients' healthcare provider (Cluster
  variable),
\item
  \textbf{Gender:} Subjects' Gender (0=male, 1=female),
\item
  \textbf{Age:} Subjects' age,
\item
  \textbf{Height:} Subjects' height in metres,
\item
  \textbf{Weight:} Subjects' weight in kilograms,
\item
  \textbf{BMI:} Subjects' body mass index,
\item
  \textbf{FamOb:} Family obesity history (yes or no),
\item
  \textbf{Time:} Response time in minutes (exclusion-restriction
  variable).
\end{itemize}

In this dataset, Age and FamOb are MAR, while the weight variable is
affected by selection bias, attributed to an indirect MNAR mechanism.
This MNAR mechanism typically arises when an unobserved or omitted
variable influences both the value of the incomplete variable (in this
case, Weight) and its likelihood of being missing (denoted as R).

In the primary analysis model, BMI serves as the dependent variable,
with Age, Gender, and FamOb as predictors. Because of the clustered
nature of the data, which is quantified with the Intraclass Correlation
Coefficient (ICC) below, we include random intercepts, as well as a
random slope for the Age variable. The model is represented as:
\begin{equation}
\label{eqn:main}
BMI_{ij}= (\beta_{o}+ b_{oj} ) + (\beta_{1}+ b_{oj})* Age_{ij} + \beta_{2}*FamOb_{ij}+ \beta_{3}Gender_{ij} + \epsilon_{ij}
\end{equation}

We start by loading the data:

\begin{CodeChunk}
\begin{CodeInput}
R> #data("data_heckman", package = "micemd")
R> #dat <- data_heckman
\end{CodeInput}
\end{CodeChunk}

Now, let's begin by examining the missing patterns in the data by
cluster:

\begin{CodeChunk}
\begin{CodeInput}
R> library(ggpubr)
R> myplots <- lapply(1:5, function(i) {
+   ggmice::plot_pattern(setDT(Obesity)[Cluster==i])+
+   ggplot2::ggtitle(paste0("Cluster", i))
+  })
R> ggarrange(myplots[[1]], myplots[[3]], nrow=1,common.legend = TRUE, legend="bottom")
\end{CodeInput}
\begin{figure}

{\centering \includegraphics[width=0.7\linewidth]{Imputation_of_Incomplete_Multilevel_Data_files/figure-latex/obesity-md-1} 

}

\caption[Missing pattern]{Missing pattern}\label{fig:obesity-md}
\end{figure}
\end{CodeChunk}

We observe that the missing pattern is non-monotonic and quite similar
across the clusters. However, regarding the weight variable, we notice
that is systematically missing in cluster 3. In order to evaluate if we
require a imputation method that accounts for clustering we assess the
Intraclass Correlation

\begin{CodeChunk}
\begin{CodeInput}
R> Nulmodel <- lme4::lmer(BMI ~ 1 + (1|Cluster), data = Obesity)
R> #Performance::ICC(Nulmodel)
\end{CodeInput}
\end{CodeChunk}

Since the ICC is above 0.1 and as the main analysis will be use a mixed
model, we decide to use two-level (2l) imputation methods. In this
imputation process, we include all predictor variables from equation
\ref{eqn:main} in the main model. However, since BMI is a composite of
weight and height, we use deterministic imputation for these, which is
described below.

We use the \textbf{find.defaultMethodfunction} provided in the
\textbf{micemd} package, which suggests an appropriate method for MAR
variables based on the type of variable, number of observations in the
cluster, and number of clusters.

It suggests using `2l.2stage.bin' for the FAV variable and
`2l.2stage.norm' for the age variable. However, after inspecting the age
density plot, we consider modifying its method to `2l.2stage.pmm'. For
the BMI variable, we employ deterministic imputation.

\begin{CodeChunk}
\begin{CodeInput}
R> meth_mar <- find.defaultMethod(Obesity, ind.clust=1, I.small = 7,
+                                ni.small = 100, prop.small = 0.4)
R> meth_mar["BMI"]<- "~ I(Weight / (Height)^2)"
R> meth_mar["Age"]<-"2l.2stage.pmm" 
\end{CodeInput}
\end{CodeChunk}

For these imputation models, it is necessary to specify the prediction
matrix, with the cluster variable labelled as -2 and the predictor
variable measured within clusters labelled as 2, encompassing all
variables. We need to suprime the variable Time as this variable is not
specified in the main model. We also modify the relationship between
BMI, weight and height in the prediction matrix to avoid circular
predictions. Then we proceed to run the imputation model.

\begin{CodeChunk}
\begin{CodeInput}
R> pred_mar <- quickpred(Obesity)   # predictor matrix
R> pred_mar[,"Cluster"] <- -2 # clustering variable
R> pred_mar[,"Time"] <- 0
R> pred_mar[pred_mar==1] <- 2
R> pred_mar[c("Height", "Weight"), "BMI"] <- 0
R> ggmice::plot_pred(pred_mar)
\end{CodeInput}


\begin{center}\includegraphics[width=0.7\linewidth]{Imputation_of_Incomplete_Multilevel_Data_files/figure-latex/obesity-predmar-1} \end{center}

\begin{CodeInput}
R> imp_mar <- mice(data = Obesity, meth = meth_mar, pred = pred_mar,
+                 m=10, seed = 123, printFlag = FALSE)
\end{CodeInput}
\end{CodeChunk}

\begin{CodeChunk}
\begin{CodeInput}
R> summary(complete(imp_mar,"long")$Weight)
\end{CodeInput}
\begin{CodeOutput}
   Min. 1st Qu.  Median    Mean 3rd Qu.    Max. 
 -9.826  69.500  82.789  82.337  94.919 148.712 
\end{CodeOutput}
\end{CodeChunk}

We are also contemplating the utilisation of the ???pmm??? option, as
the values imputed using a fully parametric method may be implausibly
low for some patients.

\begin{CodeChunk}
\begin{CodeInput}
R> meth_mar["Weight"]<-"2l.2stage.pmm" 
R> imp_mar_pmm <- mice(data = Obesity, meth = meth_mar, pred = pred_mar,
+                     m=10, seed = 123, printFlag = FALSE)
\end{CodeInput}
\end{CodeChunk}

\begin{CodeChunk}
\begin{CodeInput}
R> summary(complete(imp_mar_pmm,"long")$Weight)
\end{CodeInput}
\begin{CodeOutput}
   Min. 1st Qu.  Median    Mean 3rd Qu.    Max. 
  28.35   67.80   80.53   80.90   92.94  134.61 
\end{CodeOutput}
\begin{CodeInput}
R> #ggmice::ggplot_trace(imp_mar_pmm, "Weight")
\end{CodeInput}
\end{CodeChunk}

After confirming convergence, we proceed to save the results for future
use. We consider the possibility that patients may not have been
selected randomly, which would then have led to a distribution for
weight that does not reflect the weight in the population. It???s likely
that an omitted variable, like self-esteem, could influence this
selection. For instance, individuals with lower self-esteem might have
higher weight values, impacting their willingness to provide honest
information due to embarrassment.

To address this situation, two approaches have been proposed for dealing
with Missing Not at Random (MNAR) data: pattern-mixed models and
selection models. Within pattern-mixed models, methods like the delta
method and more advanced ones like NARFS have been suggested. The
selection model approach includes methods such as the Heckman model,
which can be particularly useful in this case. Several methods,
including those by \cite{Galimar_2017,Hammon_2021}, and the recently a
Heckman method designed for two-level data, allow for variations in
intercepts and exposure effects (random intercept and slope).

To apply the \textbf{2l.2stage.heckman} method, the weight variable
should be specified as `2l.2stage.heckman' found in the micemd package.
Additionally, the prediction matrix needs modification because this
method involves specifying two equations: one for the outcome,
describing the incomplete variable in terms of partially observed
predictors (in this case, all variables from the main model), and the
other for the selection model, explaining the probability of being
observed based (R) on certain variables. For the outcome equation we
consider the same imputation model that we used for the MAR case (main
model).
\[Weight_{ij}= \beta^O_{o} + \beta^O_{1}Age_{ij} + \beta^O_{2}FamOb_{ij}+ \beta^O_{3}Gender_{ij} + \epsilon^O_{ij}\]
Regarding the selection equation, we include the same predictors as
those in the main model, as well as a time variable. Here the time
variable serves as a restriction exclusion variable specifically
explaining the probability of being observed but not affecting the
incomplete value (Weight). In this context, we assume that the time a
user spends completing the survey serves as a proxy for the barriers
they may encounter in survey completion, such as familiarity with the
survey content or internet speed. These factors may lead the user to
skip specific questions or even the entire survey. Also, we assume the
time does not have any influence on the subject???s weight.
\[R_{ij}= \beta^S_{o} + \beta^S_{1}Age_{ij} + \beta^S_{2}FamOb_{ij}+ \beta^S_{3}Gender_{ij} +\beta^S_{4}Time_{ij}+ \epsilon^S_{ij}\]

These two equations are jointly estimated under the assumption that the
error terms are interconnected with a bivariate normal distribution. For
a more comprehensive understanding of the model and the exclusion
restriction, see \cite{Munoz_2022}.

To use information from both equations, we must adjust the prediction
matrix. The cluster variable remains specified as before (-2). In this
imputation method, all the variables present in both the selection and
outcome equations are included with a random effect.

However, it is essential to distinguish which of these variables appear
in each equation. In this framework, when a variable is shared between
both equations, it is denoted as (2). Predictors exclusive to the
outcome equation are indicated as (-4), while those exclusive to the
selection equation are labelled as (-3). Consequently, the only
alteration needed in the predictor matrix pertains to the variable
`Time'.

\begin{CodeChunk}
\begin{CodeInput}
R> pred_mnar <- pred_mar
R> pred_mnar["Weight","Time"]<- -3
R> #ggmice::plot_pred(pred_mnar)
\end{CodeInput}
\end{CodeChunk}

We also need to modify the method of the weight variable.

\begin{CodeChunk}
\begin{CodeInput}
R> meth_mnar <- meth_mar
R> meth_mnar["Weight"]<- "3l.2stage.heckman"
\end{CodeInput}
\end{CodeChunk}

Then we proceed to run the imputation model as before, after executing
these imputation procedures, it is essential to assess convergence and
the coherence of the imputed values.

\begin{CodeChunk}
\begin{CodeInput}
R> imp_mnar<- mice(data = Obesity, meth = meth_mnar, pred = pred_mnar,
+                 m=10, seed = 123, printFlag = FALSE)
R> summary(complete(imp_mnar,"long")$Weight)
\end{CodeInput}
\begin{CodeOutput}
   Min. 1st Qu.  Median    Mean 3rd Qu.    Max. 
 -7.334  69.553  83.722  83.086  96.738 183.338 
\end{CodeOutput}
\end{CodeChunk}

Upon examining the weight variable, we noticed that the imputed range
falls outside the realm of plausible values (as weight should be
positive).

\begin{CodeChunk}
\begin{CodeInput}
R> summary(complete(imp_mnar,"long")$Weight)
\end{CodeInput}
\begin{CodeOutput}
   Min. 1st Qu.  Median    Mean 3rd Qu.    Max. 
 -7.334  69.553  83.722  83.086  96.738 183.338 
\end{CodeOutput}
\end{CodeChunk}

Consequently, as before we use the `pmm', option but this time for the
Heckman imputation, this approach ensures that the imputed values remain
within the range of observable values. We then run the imputation model
but this time using the option of pmm, to assure that weight values are
in the range of the observable data, this can be implemented by setting
the pmm parameter to true.

\begin{CodeChunk}
\begin{CodeInput}
R> imp_mnar_pmm <- mice(data = Obesity, meth = meth_mnar, pred = pred_mnar,
+                      m=10, seed = 123, pmm = T,  printFlag = FALSE)
\end{CodeInput}
\end{CodeChunk}

We check the convergency of the results

\begin{CodeChunk}
\begin{CodeInput}
R> summary(complete(imp_mnar_pmm,"long")$Weight)
\end{CodeInput}
\begin{CodeOutput}
   Min. 1st Qu.  Median    Mean 3rd Qu.    Max. 
  28.35   71.68   85.07   85.07   98.21  134.61 
\end{CodeOutput}
\begin{CodeInput}
R> #ggmice::ggplot_trace(imp_mar_pmm, "Weight")
\end{CodeInput}
\end{CodeChunk}

After this modification we proceed to compare the effects on the model.
We run the analysis model on each of the completed datasets as well
asthe dataset where the incomplete values are removed (Complete Case
analysis, CC).

\begin{CodeChunk}
\begin{CodeInput}
R> library(ggplot2)
R> cc_rs<- with(setDT(Obesity)[complete.cases(Obesity),],
+              lme( BMI ~ Age + FamOb + Gender, random=~1+Age|Cluster))
R> mar_rs <- with(imp_mar,lme( BMI ~ Age + FamOb + Gender,random=~1+Age|Cluster))
R> mar_pmm_rs <- with(imp_mar_pmm,lme( BMI ~ Age + FamOb + Gender,random=~1+Age|Cluster))
R> mnar_rs<- with(imp_mnar,lme(BMI ~ Age + FamOb + Gender,random=~1+Age|Cluster))
R> mnar_pmm_rs<- with(imp_mnar_pmm, lme(BMI ~ Age + FamOb + Gender,random=~1+Age|Cluster))
R> 
R> list_models<-list(cc_rs,mar_rs,mar_pmm_rs,mnar_rs,mnar_pmm_rs)
R> plot_models(list_models,
+             mod_name = c("Complete case", "MAR","MAR_pmm", "MNAR", "MNAR_pmm"))
\end{CodeInput}


\begin{center}\includegraphics{Imputation_of_Incomplete_Multilevel_Data_files/figure-latex/models-1} \end{center}

\end{CodeChunk}

We note that there is minimal disparity in the age effect, FamObs, or
Gender across the various imputation models under consideration. An
analysis of the intercept reveals that, under the MNAR assumption, a
higher average BMI is anticipated compared to the MAR assumption.
Nonetheless, with respect to precision of estimates, we notice that in
general MNAR imputation leads to wider confidence intervals, in this
case it does not have any influence on the final result but there could
be cases where variation in the assumed missing mechanism could lead
also to differences on significant test and therefore lead to
contradictory conclusions.

\hypertarget{additional}{%
\section{Additional}\label{additional}}

The imputation of these data is based on the
\href{https://github.com/johamunoz/Heckman-IPDMA/blob/main/Toy_example.R}{IPDMA
Heckman Github repo}

Visualize missing data pattern:

The matrix only shows the predictors for the main model, not the
selection model.

\hypertarget{discussion}{%
\section{Discussion}\label{discussion}}

ORDER:

\begin{itemize}
\item
  summary
\item
  congeniality, then in hierarchical models
\item
  look whether we can fit cong. back in the main body
\item
  alt. methods
\item
  conclusion: mice is really easy!
\item
  Additional levels of clustering
\item
  More complex data types: timeseries and polynomial relationship in the
  clustering.
\item
  FIML vs MI
\end{itemize}

An alternative approach to missing data is to use Full Information
Maximum Likelihood (FIML). This method does not require the imputation
of any missing values. Whereas MI consists of imputation, analyses and
pooling steps, FIML analyses the data in a single step. When the
assumptions are met the two approaches should produce equivalent
results. {[}REF{]} As FIML requires specialised software, not all
analyses can be performed with standard software. {[}REF{]}

\begin{itemize}
\tightlist
\item
  Survival / TTE, this could be put in the paragraph on congeniality
\end{itemize}

When the outcome is time-to-event, the Nelson-Aalen estimate of the time
to event should be included as a covariate in the imputation model
{[}REF{]}

\hypertarget{funding}{%
\section{Funding}\label{funding}}

This project has received funding from the European Union's Horizon 2020
research and innovation programme under ReCoDID grant agreement No
825746.

The views expressed in this paper are the personal views of the authors
and may not be understood or quoted as being made on behalf of or
reflecting the position of the regulatory agency/agencies or
organizations with which the authors are employed/affiliated.

\hypertarget{references}{%
\section{References}\label{references}}

\hypertarget{appendix}{%
\section{Appendix}\label{appendix}}

Table 3: Notation

\begin{longtable}[]{@{}
  >{\raggedright\arraybackslash}p{(\columnwidth - 2\tabcolsep) * \real{0.5417}}
  >{\raggedright\arraybackslash}p{(\columnwidth - 2\tabcolsep) * \real{0.4583}}@{}}
\toprule\noalign{}
\begin{minipage}[b]{\linewidth}\raggedright
\textbf{Formula} \pkg{lme4}
\end{minipage} & \begin{minipage}[b]{\linewidth}\raggedright
\textbf{Details}
\end{minipage} \\
\midrule\noalign{}
\endhead
\bottomrule\noalign{}
\endlastfoot
\texttt{y\ \textasciitilde{}\ x1\ +\ (1\ \textbar{}\ g1)} & Fixed
\texttt{x1} predictor with random intercept \\
& varying among \texttt{g1} \\
\texttt{y\textasciitilde{}x1*x2+\ (1\textbar{}\ g1)} & Interactions of
\texttt{x1} and \texttt{x2} only in fixed effect \\
\texttt{y\ \textasciitilde{}\ x1*x2+\ (x2\textbar{}\ g1)} & Interactions
of \texttt{x1} and \texttt{x2} only in fixed effect \\
& with slope of \texttt{x2} randomly varying among \texttt{g1} \\
\texttt{y\ \textasciitilde{}\ x1*x2+\ (x1*x2\textbar{}\ g1)} &
variance-covariance matrix estimated only with \\
& the variance terms of intercept, slope of \texttt{x1}, \\
& slope of \texttt{x2} and interaction \texttt{x1*x2} \\
\texttt{y\ \textasciitilde{}\ x1*x2+\ (x1\ \textbar{}\ g1)+\ (x2\textbar{}\ g1)}
& variance-covariance matrix estimated separately, \\
& i.e, one for intercept and \texttt{x1} and another for \\
& intercept and \texttt{x2} \\
\texttt{y\ \textasciitilde{}\ x1\ +\ (x1\ \textbar{}\ g1)\ or\ 1\ +\ x1\ +\ (1\ +\ x1\ \textbar{}\ g1)}
& Fixed \texttt{x1} with correlated random intercept and \\
& random slope of \texttt{x} \\
\texttt{y\ \textasciitilde{}\ x1\ +\ (x1\ \textbar{}\textbar{}\ g1)\ or\ 1\ +\ x1\ +\ (1\ \textbar{}\ g1)\ +\ (0\ +\ x1\ \textbar{}\ g1)}
& Fixed \texttt{x1} with uncorrelated random intercept \\
& and random slope of \texttt{x1} \\
\texttt{y\ \textasciitilde{}\ (1\ \textbar{}\ g1)\ +\ (1\ \textbar{}\ g2)}
& Random intercept varying among \texttt{g1} and among
\texttt{g2\ \ \ \textbar{}\ \textbar{}}y \textasciitilde{} (1 \\
& \\
\end{longtable}

\bibliography{../References/multilevelmice.bib}



\end{document}
